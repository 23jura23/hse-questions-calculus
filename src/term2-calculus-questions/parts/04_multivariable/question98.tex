\Subsection{Билет 98: Неравенство Адамара.}

Еще один пример для метода множителей Лагранжа.

\begin{example}[Неравенство Адамара]\slashns
    
    Дана матрица $A$, суммы квадратов строк равны $h_1^2, h_2^2, \ldots, h_n^2$. Тогда $\abs{\det A} \leqslant h_1h_2\ldots h_n$.
\end{example}

\begin{proof}\slashns
    \[
    A = \begin{pmatrix}
    x_{11} & \cdots & x_{1n} \\
    \vdots &        & \vdots \\
    x_{n1} & \cdots & x_{nn} \\
    \end{pmatrix}
    \]
    
    Запишем условия для матрицы $A$
    
    $\Phi_k(x) = \sum\limits_{j=1}^n x_{kj}^2 - h_k^2 = 0$
    
    Функция которую мы хотим максимизировать $f(x) = \det A$
    
    Поймем, что экстремум существует. $\Phi_k(x)$ -- сфера в $n$ мерном пространстве, а это компактное множество, а значит все условия тоже компакт. $f(x)$ -- непрерывная функция на компакте, значит есть минимум и максимум.
    
    $F(x) = \det - \sum\limits_{k=1}^n \lambda_k \Phi_k(x)$.
    
    $\dfrac{\partial F}{\partial x_{ij}} = 0$
    
    Разберемся, что такое $\dfrac{\partial \Phi_k}{\partial x_{ij}}$
    
    $\dfrac{\partial \Phi_k}{\partial x_{ij}} = 0$, если $i \not = k$
    
    $\dfrac{\partial \Phi_k}{\partial x_{ij}} = 2x_{ij}$, если $i = k$
    
    Разберемся, что такое $\dfrac{\partial \det}{\partial x_{ij}}$
    
    $\dfrac{\partial \det}{\partial x_{ij}} = \dfrac{\partial}{\partial x_{ij}} \left( \sum\limits_{k=1}^{n} x_{ik}\cdot A_{ik} \right) = A_{ij}$, где $A_{ik}$ -- алгебраические дополнения. 
    
    $0 = \dfrac{\partial F}{\partial x_{ij}} = A_{ij} - \lambda_i \cdot 2x_{ij} \Rightarrow A_{ij} = 2\lambda_i x_{ij} \Rightarrow \det = \sum\limits_{k=1}^n x_{ik}A_{ik} = \sum\limits_{k=1}^n x_{ik} 2\lambda_i x_{ik} = 2\lambda_i h_i^2 \Rightarrow \lambda_i = \dfrac{\det}{2h_i^2} \Rightarrow A_{ij} = \dfrac{\det}{h_{ij}} x_{ij} \Rightarrow \dfrac{A_{ij}}{\det} = \dfrac{x_{ij}}{h_i^2}$
    
    Заметим, что наше уравнение можно доказать, только при $h_i = 1$, т.к. если умножить строку на $t$, то и определитель увеличится в $t$ раз, тогда если мы все строчки разделим на их длины, то все длины будут равны 1, а неравенство не изменится.
    
    $\Rightarrow \dfrac{A_{ij}}{\det} = x_{ij} \Rightarrow (A^{-1})^T = A \Rightarrow A^{-1} = A^T$
    
    $1 = \det AA^{-1} = \det A \det A^{-1} = \det A \det A^T = (\det A)^2 \Rightarrow \det A = \pm 1$
    
    Для экстремумов верно, значит верно и для всех.
    
\end{proof}