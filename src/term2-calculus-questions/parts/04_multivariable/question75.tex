\Subsection{Билет 75: Производная по направлению. Экстремальное свойство градиента}


	
\begin{definition}[Направление] \thmslashn
		
	Это такой вектор, единичной длины, который смотрит в нужную нам сторону.  	
		
\end{definition}

\begin{definition}[Производная по направлению] \thmslashn
	
	Имеем направление $ h $. Также есть внутренняя точка $ a $, то есть $ a \in \mathrm{Int} E $. И сама функция $ f : E \mapsto \mathbb{R} $. Напомню, что $ E \subset \mathbb{R}^n $. 
	
	Тогда следующая штука, это производная функции $ f $ по направлению $ h $ в точке $ a $:
	\[
	\frac{\partial f}{\partial h} (a) := \lim_{t \to 0} \frac{f(a + t \cdot h) - f(a)}{t}
	\]

\end{definition}

\begin{theorem}[Вспомогательная теорема] \thmslashn
	
	Имеем направление $ h $, $ f : E \mapsto \mathbb{R} $, $ a \in \mathrm{Int} E $ и $ f $ дифференцируема в точке $ a $.
	
	Тогда выполняются следующие равенства:
	\[
	\frac{\partial f}{\partial h} (a) = d_a f (h) = \langle \nabla f, h \rangle
	\]
	
	\begin{proof} \thmslashn
			
		Воспользуемся определением дифференцируемости функции $ f $: 
		\[
		f(a + t \cdot h) = f(a) + T(t \cdot h) + \alpha(t \cdot h)
		\]
		
		Воспользуемся линейностью и получим:
		\[
		f(a + t \cdot h) = f(a) + t \cdot Th + \alpha(t \cdot h) \implies
		\]
		\[
		\implies f(a + t \cdot h) - f(a) = t \cdot Th + \alpha(t \cdot h)
		\]
		
		Теперь распишем производную по направлению
		\[
		\frac{\partial f}{\partial h} (a) = \lim_{t \to 0} \frac{f(a + t \cdot h) - f(a)}{t}
		\]
		
		Благодаря прошлым замечанием, можем заменить числитель:
		\[
		\frac{\partial f}{\partial h} (a) = \lim_{t \to 0} \frac{t \cdot Th + \alpha(t \cdot h)}{t} = \lim_{t \to 0} (Th + \frac{\alpha(t \cdot h)}{t})
		\]
		
		Заметим, что $ Th $ какая-то константа, поэтому можно вынести:
		\[
		\frac{\partial f}{\partial h} (a) = Th + \lim_{t \to 0} \frac{\alpha(t \cdot h)}{t}
		\]
		
		Заметим, что $ \| t \cdot h \| = t \cdot \| h \| = t $: норма направления равна единице.
		Следовательно $ \lim_{t \to 0} \frac{\alpha(t \cdot h)}{t} = \lim_{t \to 0} \frac{\alpha(t \cdot h)}{\| t \cdot h \|} = 0 $: по определению $ \alpha $.  \\
		Получаем, что
		\[
		\frac{\partial f}{\partial h} (a) = Th
		\]
		
		Осталось вспомнить, что $ Th = d_a f (h) $, просто альтернативная запись. Следовательно, первое равенство у нас есть. \\ \\
		Разберёмся со вторым. Но тут совсем всё просто, так как перед нами определение градиента. ()
		
	\end{proof}
	
\end{theorem}

\begin{consequence}[Экстремальное свойство градиента] \thmslashn
	
	Имеем направление $ h $, $ f : E \mapsto \mathbb{R} $, функция $ f $ дифференцируема в точке $ a $ и $ \nabla f(a) \neq 0 $.
	
	Тогда для любого направления $ h $ выполнено следующее: 
	\[
	- \| \nabla f(a) \| \leq \frac{\partial f}{\partial h} (a) \leq \| \nabla f(a) \|
	\]
	
	А равенство достигается, когда $ h = \pm \frac{\nabla f(a)}{\|\nabla f(a) \|} $
	
	\leavevmode \linebreak
	\textbf{Смысл следствия.} Оно объясняет физический смысл градиента. Градиент это вектор, в направлении которого функция меняется быстрее всего.
	
	\begin{proof} \thmslashn
		
		Напишем равенство, которое мы вывели в прошлой теореме:
		\[
		\frac{\partial f}{\partial h} (a) = \langle \nabla f, h \rangle \implies
		| \frac{\partial f}{\partial h} (a) | = | \langle \nabla f, h \rangle |
		\]
		
		Напишем неравенство Коши-Буняковского. 
		\[
		| \langle \nabla f, h \rangle | \leq \| \nabla f(a) \| \cdot \| h \| = \| \nabla f(a) \| \implies
		| \frac{\partial f}{\partial h} (a) | \leq \| \nabla f(a) \|        
		\] 
		
		То есть неравенство мы уже доказали. 
		
		Далее осталось вспомнить, когда в неравенстве Коши-Буняковского, получается равенство. Это происходит тогда и только тогда, когда вектора являются пропорциональными. Осталось вспомнить, что норма $ h $ должна быть строго равна единице. Следовательно у нас есть два возможных выбора для $ h $:
		\[
		h = \pm \frac{\nabla f(a)}{\|\nabla f(a) \|}
		\]
		
	\end{proof}
	
\end{consequence}






