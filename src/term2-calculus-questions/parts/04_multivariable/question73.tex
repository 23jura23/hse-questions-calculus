\Subsection{Билет 73: Дифференцируемость отображений из $\mathbb{R}^n$ в $\mathbb{R}^m$. Частные случаи. Матрица Якоби. Градиент.}
\begin{definition} \thmslashn

    $f: E \mapsto \mathbb{R}^m$ \ \ \ $a \in \Int E, E \subset \mathbb{R}^n$

    $f$ - дифференцируема в точке $a$, если существует линейное отображение $T: \mathbb{R}^n \mapsto \mathbb{R}^m$,

    такое что $f(a + h) = f(a) + Th + \alpha(h)$, где $\dfrac{\alpha(h)}{||h||} \rightarrow 0$ при $h \rightarrow 0$

\end{definition}

\begin{remark} \thmslashn

    Заметим, что в нашем определении все 0 - векторы, ровно как и аргументы. 

    Например, $h \in \mathbb{R}^n$, $\alpha(h) \in \mathbb{R}^m$

\end{remark}

\begin{remark} \thmslashn

    Несложно убедиться, что данное определение ровно такое же, как и определение,

    которое давалось, когда мы говорили про дифференцируемость функции

    одной переменной. С той лишь разницей, что тогда вместо $T$  у нас было 

    просто домножение на константу (тоже линейное отображение, но тривиальное),

    а добавкой была $o(||h||)$ (но у нас записано тоже самое, ведь по сути $\alpha(h) = o(||h||)$). 

    Получается, что дифференцируемость функции от одной переменной,

    про которую мы говорили раньше - это частный случай при $n = m = 1$.

\end{remark}

\begin{definition} \thmslashn

    $T$ - дифференциал функции $f$ в точке $a$. Обозначается чаще всего $d_af$

\end{definition}

\begin{remark} \thmslashn

    Если $T$ существует, то оно определено однозначно. 

    Зафиксируем $h \in R^n$. $f(a + th) = f(a) + T(th) + \alpha(th)$, где $t \in R$

    Так как $T$ - линейно, то $T(th) = t T(h)$

    $Th = \dfrac{f(a + th) - f(a)}{t} - \dfrac{\alpha(th)}{t}$ Перейдем к пределу при $t \to 0$

    Так можно сделать, потому что $\dfrac{\alpha(th)}{t} \to 0$ при $t \to 0$,

    поскольку это записано в определении дифференцируемости функции (фиксированное $h$).

    Тогда получили:

    $\lim\limits_{t \to 0} \dfrac{f(a + th) - f(a)}{t} = Th$

    Значит это отображение однозначно.
\end{remark}

\begin{definition} \thmslashn

    Матрица линейного оператора $T$ - матрица Якоби функции $f$ в точке $a$

    Обозначается матрица T: $f'(a)$

    Данное обозначение намекает, что эта матрица - некий аналог производной

    для функции одной переменной
   
\end{definition}

\begin{remark} \thmslashn

    Дифференцируемость функции $f$ в точке $a$ влечет непрерывность $f$ в точке $a$. 

    $f(a + h) = f(a) + Th + \alpha(h)$. 

    Перейдем к пределу при $h \to 0$, получим: 

    $f(a) + Th + \alpha(h) \to f(a) + 0 + 0 = f(a)$,

    так как $\alpha(h)$ при делении на $||h||$ уже будет стремится к 0, здесь же тем более

    Получили определение непрерывности

\end{remark}

\begin{example}[Важный частный случай $m = 1$] \thmslashn
    
    Получаем отображение $f: E \mapsto \mathbb{R}, E \subset \mathbb{R}^n$

    Мы из вектора сделали число. Это скалярное произведение на какой-то вектор.

    Потому что можно представить, что умножаем матрицу на вектор и получаем вектор

    размера 1.

    Откуда получаем, что эта матрица - это строчка размера $n$.

    Ну а это - скалярное произведение.

    $f(a + h) = f(a) + \langle v, h \rangle + \alpha(h)$ для некоторого $v \in R^n$

\end{example}

\begin{definition} \thmslashn

$v$ - градиент функции $f$ в точке $a$

Обозначается: $\grad f$ или $\nabla f$ 

\end{definition}
