\Subsection{Билет 84: Мультииндексы. Определения, обозначения, лемма о производной композиции гладкой и линейной функций.}

\begin{definition}[Мультииндекс] \thmslashn


$k = (k_1, k_2,..., k_n) \quad k_j \geq 0 \quad k_j \in \mathbb{Z}$

Высота мультииндекса $|k| := k_1+k_2+k_3+...+k_n$

$k! := k_1!k_2!...k_n!$

$h \in \mathbb{R}^n \quad h^k := h_1^{k_1}h_2^{k_2}...h_n^{k_n}$

$f^{(k)} := \frac{\partial^{|k|}f}{\partial x_1^{k_1} \partial x_2^{k_2}...\partial x_n^{k_n}}$

Полиномиальный или мультиномиальный коэффициент:

$\binom{|k|}{k_1, k_2, ..., k_n} := \frac{|k|!}{k!} = \frac{(k_1+k_2+...+k_n)!}{k_1!k_2!...k_n!}$ -- количество способов покрасить $|k|$ шаров в $n$ цветов так, чтобы первого цвета было $k_1, k_2$ --  второго и т.д.

\end{definition}

\begin{lemma} \thmslashn

$f: D \to \mathbb{R} \;\; f\in C^r(D) \;\; D \subset \mathbb{R}^n$

$[x, x+h]$ -- отрезок с концами $x$ и $x+h$. (на многомерном пространстве)

$[x, x+h] \subset D$

$F(t) = f(x+th) \;\; F:[0,1] \to \mathbb{R}$ (функция от одной переменной)

Тогда $F\in C^r[0,1]$ и при $0\le l \le r$

$F^{(l)}(t) = \sum\limits_{|k| = l} \binom{l}{k_1, k_2,...,k_n}f^{(k)} (x+th) h^k$

\begin{proof} \thmslashn

$G(t) := g(x+th)$

$G'(t) = g'(x+th)\cdot(x+th)' = (\frac{\partial g}{\partial x_1}(x+th), ..., \frac{\partial g}{\partial x_n}(x+th))\cdot \begin{pmatrix}
h_1\\h_2\\\vdots\\h_n
\end{pmatrix} = 
\sum\limits_{i = 1}^{n} \frac{\partial g}{\partial x_i}(x+th) h_i$\\ -- получили формулу для $l = 1$.

$F''(t) = (F'(t))' = (\sum\limits_{i = 1}^{n} \frac{\partial f}{\partial x_i}(x+th) h_i)' = \sum\limits_{i = 1}^{n}\sum\limits_{j = 1}^{n} \frac{\partial^2 f}{\partial x_j \partial x_i}(x+th) h_jh_i$

$F^{(l)} (t) = \sum\limits_{i_{l} = 1}^{n}\sum\limits_{i_{l-1} = 1}^{n}...\sum\limits_{i_1 = 1}^{n} \frac{\partial^l f}{\partial x_{i_l} \partial x_{i_{l-1}}...\partial x_{i_1}} (x+th)h_{i_1}h_{i_2}...h_{i_n} = \sum\limits_{|k| = l} \binom{|k|}{k_1,k_2,...,k_n}f^{(k)}(x+th)h^k =\\= \sum\limits_{|k| = l} \frac{|k|!}{k!}f^{(k)}(x+th)h^k$ (можем сделать замену на $f^{(k)}$ по теореме из билета 83)

$k = (\#\{j: i_j = 1\}, \#\{j: i_j = 2 \},...)$ \quad ($\#$ -- кол-во чего-то)

\end{proof}
\end{lemma}