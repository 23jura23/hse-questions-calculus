\Subsection{Билет 87: Теорема Банаха о сжатии. Следствие. Метод касательных для решения уравнения}

\begin{theorem}[Теорема Банаха о сжатии] \thmslashn
	
	X – полное метрическое пространство. $f: X \mapsto X$,\; $0 < \lambda < 1$ и $\rho(f(x),f(y)) \le \lambda \rho(x, y)$ $\forall x, y \in X$. 
	
	Тогда существует единственная неподвижная точка, такая что $f(x) = x$.
	\begin{proof} \thmslashn
		
		\begin{itemize}
			\item Единственность.
			От противного. Пусть неподвижных точек две: $\widetilde{x}$ и $x$. Тогда $\rho(x,\;\widetilde{x})$ = $\rho(f(x),\;f(\widetilde{x}))$ $\le$
			$\lambda \rho(x,\;\widetilde{x})$. Но $\lambda < 1$. Противоречие.
			
			\item Существование.
			
			Возьмем произвольную начальную точку $x_0 \in X$ и $x_{n+1} = f(x_n)$. Докажем, что это фундаментальная последовательность.
			
			\[\rho(x_n,\;x_{n + k}) = \rho(f(x_{n-1}),\;f(x_{n - 1 + k})) \le \lambda \rho(x_{n-1},\; x_{n-1+k}) \le ... \le \lambda^n \rho(x_0, x_k)\]
			
			Попытаемся оценить $\rho(x_0, x_k)$ по неравенству треугольника.
			\[
			\rho(x_{0},\; x_{k}) \le \rho(x_{0},\; x_{1}) + \rho(x_{1},\; x_{2}) + ... + \rho(x_{k-1},\; x_{k}) \le \rho(x_{0},\; x_{1}) + \lambda \rho(x_{0},\; x_{1}) + ... + \lambda^{k-1}\rho(x_{0},\; x_{1})
			\]
			
			А это убывающая геометрическая прогрессия. Тогда, 
			
			$
			\rho(x_{0},\; x_{k}) < \frac{\rho(x_{0},\; x_{1})}{1 - \lambda}
			$. Вернемся к $\rho(x_n,\;x_{n + k})$ . Теперь мы можем это оценить:
			
			\[
			\rho(x_n,\;x_{n + k}) < \lambda^n \frac{\rho(x_{0},\; x_{1})}{1 - \lambda} \longrightarrow 0
			\]
			
			Значит, рассматриваемая последовательность фундаментальна. Значит, $\exists \lim_{n\to \infty} x_n =: x^*  $.
			
			\[
			f(x^*) = f(\lim_{n\to \infty} x_n) = \lim_{n\to \infty} f(x_n) = \lim_{n\to \infty} x_{n+1} = x^* 
			\] 
			
			По непрерывности функции f. Откуда непрерывность? Рассмотрим: $\rho(f(x),\;f(y)) \le \lambda \rho(x, y)$ $\forall x, y \in X$. $f$ это функция, уменьшающая расстояния. Поэтому, если $y \to x$, то $\rho (x,\; y) \to 0$. Тогда и $f(y) \to f(x)$. 
			Значит, $x^*$ и есть неподвижная точка. Что и требовалось доказать.
		\end{itemize}
	\end{proof}
	
\end{theorem}

\begin{statement}\thmslashn
	
	\[
	\rho(x_n, x^*) \le \lambda^n \frac{\rho(x_0, x_1)}{1 - \lambda}
	\]
	\begin{proof} \thmslashn
		
		Это следует из $\rho(x_n,\;x_{n + k}) < \lambda^n \frac{\rho(x_{0},\; x_{1})}{1 - \lambda}$. Возьмем и устремим $k$ к бесконечности.
	\end{proof}
\end{statement}

\begin{consequence} \thmslashn
	
	$X$ - полное метрическое пространство, $f, g: X \mapsto X$ - сжатия с коэф. $\lambda \in (0, 1)$. $x = f(x)$ и $y = g(y)$ - неподвижные точки.
	
	Тогда $\rho(x,\; y) \le \frac{\rho(f(x),\; g(x))}{1- \lambda}$
	\begin{proof} \thmslashn
		
		\[
		\rho(x,\;y) = \rho(f(x),\; g(y)) \le \rho(f(x),\; g(x)) + \rho(g(x),\; g(y)) \le \rho(f(x), g(x)) + \lambda \rho(x, y). 
		\]
		
		Добавили и вычли $g(x)$, расскрыли по нер-ву треугольника, оценили расстояние через сжатие, получили то, что нам нужно.
	\end{proof}
\end{consequence}

\begin{example}[Метод касательных (метод Ньютона)] \thmslashn
	
	$f \in C^2[a, x_0]$, $f'(a) =: \mu > 0$, $f(a) = 0$ и $f$ строго выпукла и строго монотонна. Хотим найти корень функции(быстрее чем бинпоиск).
	
	Рассмотрим вспомогательную функцию $g(x) := x - \frac{f(x)}{f^\prime(x)} : [a, x_0] \mapsto [a, x_0]$.
	
	Почему она действует в тот же самый отрезок?
	
	Понятно, что $g(x) \le x$, так как из $x$ мы постоянно что-то вычитаем + $f$ $f^\prime$ не отрицаительны. Более того:
	
	\[
	f(x) = f(x) - f(a) = f^\prime(c)(x - a) < f^\prime(x)(x - a), c \in [a, x]
	\]
	
	*$f(a) == 0$ + теорема Лагранжа + монотонное возрастание производной*
	
	Значит, $\frac{f(x)}{f^\prime(x)} < x - a$ $\implies$ $g(x) > a$
	
	Далее докажем, что $g$ - сжатие. Как это можно понять? По теореме Лагранжа. Мы знаем, что разница двух образов есть произведение производной в какой-либо точке t на разницу прообразов. Возьмем производную. Воспользуемся Лагранжем и тем, что производные возрастают. Пусть $M := max(f''(t))$, $t \in [a, x_0]$
	
	
	\[
	g^\prime(t) = 
	1 - \frac{f^\prime(t)f^\prime(t) - f(t) f''(t)}{(f^\prime(t))^2}
	=
	\frac{f''(t) f(t)}{(f'(t))^2} 
	<
	\frac{f''(t) f^\prime(t)(t  - a)}{(f'(t))^2}
	\]
	
	\[
	\frac{f''(t) f^\prime(t)(t  - a)}{(f'(t))^2}
	=
	\frac{f''(t)(t-a)}{f^{\prime}(t)} \le
	\frac{f''(t)(t-a)}{\mu}
	\le
	\frac{M}{\mu}(t-a)
	\le
	\frac{M}{\mu}(x_0-a)
	< 1
	\]
	
	Значит, нужно добавить в условие, что $\frac{M}{\mu}(x_0 - a) < 1$, чтобы получилось сжатие.
	
	Запустим процесс из предыдущей теоремы о сжатии. Пусть $x_n := g(x_{n - 1}) \implies \lim x_n = :x^*$ и $x^*$ - неподвижная точка.
	
	\[
	x^* = g(x^*) = x^* - \frac{f(x^*)}{f^\prime(x^*)} \implies f(x^*) = 0 \iff x^* = a
	\]
	
	Вот мы и получили способ поиска корня $a$, причем у нас есть контроль скорости.
	
\end{example}

\begin{remark} \thmslashn
	
	Откуда взялась функция $g$? Пусть $y$ - касательная графика в точке $x_0$, тогда $y = f(x_0) + f^\prime(x_0)(x - x_0)$. Рассмотрим точку, когда касательная пересечет ось абсцисс, то есть $y = 0$. Тогда  $f(x_0) + f^\prime(x_0)(x - x_0) = 0ч	 \implies x = x_0 - \frac{f(x_0)}{f^\prime(x_0)}$. А это и есть наша функция $g$.
\end{remark}