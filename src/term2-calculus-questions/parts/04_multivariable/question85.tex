\Subsection{Билет 85: ! Многомерная формула Тейлора с остатком в форме Лагранжа. Частные случаи.}

\begin{theorem} \thmslashn
	
	$D \in R^n$, $D$ - открытое множество. $f \in C^{r+1}(D)$ (функция $f$ $r + 1$ раз непр. дифференцируема на данном множестве), $[a, x] \in D$.
	
	Тогда $\exists$ $ \theta \in (0, 1) : f(x) = \sum\limits_{|k| \leq r}\dfrac{f^{(k)}(a)}{k!}(x - a)^k + \sum\limits_{|k| = r + 1}\dfrac{f^{(k)}(a + \theta(x - a))}{k!}(x - a)^k$
	
	\begin{proof} \thmslashn
		
		$F(t) := f(a + th),$ где $h = x - a$. $F \subset C^{r + 1}[0, 1]$.
		
		Запишем одномерную формулу Тейлора для $F$ в нуле.
		
		$F(t) = \sum\limits_{l = 0}^r \dfrac{F^{(l)}(0)}{l!}t^l + \dfrac{F^{(r + 1)}(\theta)}{(r + 1)!}t^{r + 1}$.
		
		Теперь, исходя из леммы из предыдущего билета, подставим все производные $F$. 
		\\Получается: $F(t) = \sum\limits_{l = 0}^r\dfrac{1}{l!} \sum\limits_{|k| = l} {l\choose{k_1, k_2, ...,k_n}} f^{(k)}(a)h^kt^l + \dfrac{1}{(r + 1)!} \sum\limits_{|k| = r + 1} {{r + 1}\choose{k_1, k_2, ..., k_n}}f^{(k)}(a + \theta h)h^k t^{r + 1}$.
		
		Осталось сделать небольшие преобразования. \\ Заметим,  что ${l\choose{k_1, k_2, ..., k_n}} = \dfrac{l!}{k!}$, ${{r + 1}\choose{k_1, k_2, ..., k_n}} = \dfrac{(r + 1)!}{k!}$, где $k$ - факториал мультииндекса.
		
		Вспомним, что нас интересует значение функции $f$ в точке $x$, это значит, что нас интересует значение $F$ в точке 1.
		
		Итого: $f(x) = F(1) = \sum\limits_{l = 0}^{r} \sum\limits_{|k| = l} \dfrac{f^{(k)}(a)}{k!}h^k + \sum\limits_{|k| = r + 1} \dfrac{f^{(k)}(a + \theta h)}{k!} h^k$.
		
		А это и есть нужная нам формула, так как такая сумма $\sum\limits_{l = 0}^{r} \sum\limits_{|k| = l} ... = \sum\limits_{|k| \leq r} ... $, а это сумма по всем мультииндексам высоты $\leq r$. Все свернулось в обещанную формулу.
	\end{proof}
\end{theorem}
\begin{example} \thmslashn
	
	Многочлен Тейлора степени r - "кусок формулы, который не остаток". \\ Выглядит он так:
	
	$\sum\limits_{|k| \leq r} \dfrac{f^{(k)}(a)}{k!}(x - a)^k$
	
\end{example}

\begin{example} \thmslashn
	
	Пусть $r = 0.$
	\\ Получим аналог теоремы Лагранжа для функции от $n$ переменных.
	
	$f(x) = f(a) + \sum\limits_{|k| = 1}\dfrac{f^{(k)}(a + \theta(x - a))}{k!}h^k$, где $h = x - a$.
	
	Мультииндекс высоты 1 - это одна единица и  остальные нули. Значит, все $k!$ под суммой = 1. \\ Производная по такому мультииндексу (пусть единица стоит на $i-$том месте) - это производная по $i-$той координате. \\ Потом мы умножаем на $h^k$, все координаты кроме $i-$той обнулятся, поэтому просто умножаем на $h_i$.
	
	Получаем такую запись: $f(x) = f(a) + \sum\limits_{i = 1}^{n}\dfrac{\delta f}{\delta x_i}(a + \theta(x - a))h_i$
	
	А это скалярное произведение градиента $f$, посчитанного в точке $(a + \theta(x - a))$ и вектора $h$.
	\\ Итого получаем $ f(x) = f(a) + \langle \triangledown f(a + \theta(x - a)), x - a \rangle$.
	
\end{example}

\begin{example} \thmslashn
	
	Пусть $n = 2$.\\
	Поймем, как будут выглядеть производные по мультииндексу, посчитанному в точке $a$. Мультииндекс для $n = 2$ будет равен $k = (i, j).$ Производная в точке $a$ = $f^{(k)} = \dfrac{\delta^{i + j}f}{\delta^ix \delta^jy}$. Подставим это в формулу.
	
	$f(x, y) = f(a, b) + \dfrac{\delta f}{\delta x}(a, b)(x - a) + \dfrac{\delta f}{\delta y}(a, b)(y - b) + \dfrac{1}{2}\dfrac{\delta^2 f}{\delta x^2}(a, b)(x - a)^2 + \dfrac{1}{2}\dfrac{\delta^2 f}{\delta y^2}(a, b)(y - b)^2 + 
	\dfrac{1}{2}\dfrac{\delta^2 f}{\delta x \delta^2 y}(a, b)(x - a)(y - b) + ...$.
	
	\textit{Расшифровка:} к значению в фиксированной точке добавляем сумму по всем мультииндексам высоты 1 - это производная по $x$ и по $y$ в точке $(a, b)$. Далее добавляем вторые производные, производную по $x$ и $y$ и так далее.
	
	Запишем производную $l-$того порядка.
	
	... + $\dfrac{1}{l!}\sum\limits_{i = 0}^{l} {l\choose i} \dfrac{\delta^{l}f}{\delta x^i \delta y^{l - i}}(a, b) (x - a)^i (y - b)^{l - i}$ + ...
	
	Данная формула даже при $n = 2$ имеет довольно много слагаемых, поэтому обычно ее используют при маленьких $r$, т.к. за счёт этого получается мало слагаемых.
	
\end{example}