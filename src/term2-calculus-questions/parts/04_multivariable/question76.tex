\Subsection{Билет 76: ! Частные производные. Элементы матрицы Якоби. Координатная запись формул для производных.}



\begin{definition}[Частная производная] \thmslashn
		
		Это просто производная по направлению от одного из базисных векторов. Определим её так:
		\[
		\frac{\partial f}{\partial x_k} (a) := \frac{\partial f}{\partial e_k} (a)
		\text{, где } e_k \text{-- k-ый базисный вектор}
		\]
		
		Давайте посмотрим на \textbf{пример}. Есть функция от двух переменных $ f(x, y) $. Хотим узнать частную производную по направлению оси абсцисса в какой-то точке $ (a, b) $. Вычислим её по определению:
		\[
		\frac{\partial f}{\partial x} (a, b) =  \lim_{t \to 0} \frac{f((a, b) - t \cdot (1, 0)) - f(a, b)}{t} = 
		\lim_{t \to 0} \frac{f(a + t, b) - f(a, b)}{t}
		\]
		
		Итак, что же здесь происходит. В данном случае мы зафиксировали второй координатный параметр. То есть функция стала зависеть только от значения первого аргумента. Иными словами, $ f $ превратилась в самую обыкновенную функцию от одной переменной, производную для которой мы отлично умеем считать. 
		
		Вывод, частная производная устроена следующим образом. Мы фиксируем все координаты кроме той, по которой мы хотим продифференцировать. В итоге рассматриваем функцию только от нужного нам аргумента, и считаем её производную. 
		
		\textbf{Еще один пример}. Дана функция $ f(x, y) = x^y $. Посчитать две частные производные. 
		\[
		\frac{\partial f}{\partial x} = y \cdot x^{y - 1}, \text{    } 
		\frac{\partial f}{\partial y} = x^y \log{x}
		\]
		
		В первом случае мы зафиксировали $ y $ -- стали воспринимать его константой. Во втором случае аналогично для $ x $.
		
\end{definition}
	
\begin{remark}[Иное определение градиента] \thmslashn
		
		Вспомним теорему из билета №75 
		\[
		\frac{\partial f}{\partial e_k} = \langle \nabla f(a), e_k \rangle \implies 
		\frac{\partial f}{\partial x_k} = \langle \nabla f(a), e_k \rangle
		\]
		
		Поймем, что представляет из себя это скалярное произведение -- это просто $ k $-ая координата вектора $ \nabla f(a) $. Отсюда делаем вывод, что градиент -- это вектор, которые состоит  из частных производных. 
		\[
		\nabla f(a) = (\frac{\partial f}{\partial x_1}, \dots, \frac{\partial f}{\partial x_n})
		\]
		
		
\end{remark}
	
\begin{remark}[Элементы матрицы Якоби] \thmslashn
		
		Пусть у нас есть $ f : E \mapsto \mathbb{R}^m $. Поймём, что такое матрица $ f'(a) $. Знаем, что она составлена из градиентов координатных функций(смотреть следствие из теоремы о дифференцируемости координатных функций -- билет №74). А мы уже знаем, как выглядит градиент(замечание выше). Поэтому матрица Якоби имеет следующий вид:
		\[
		f'(a) =
		\begin{pmatrix} 
		\frac{\partial f_1}{\partial x_1} (a) & \cdots & \frac{\partial f_1}{\partial x_n} (a) \\
		\vdots & \ddots & \vdots \\
		\frac{\partial f_m}{\partial x_1} (a) & \cdots & \frac{\partial f_m}{\partial x_n} (a) \\
		\end{pmatrix}
		\]
		
\end{remark}
	






