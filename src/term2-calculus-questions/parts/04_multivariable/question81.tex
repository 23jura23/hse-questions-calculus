\Subsection{Билет 81: Непрерывная дифференцируемость. Определение и равносильное ей свойство}

\begin{definition}\slashns
	
	$f: E \to \R^m \;\; E \subset \R^n \;\;\; a \in \Int E$
	
	$f$ -- непрерывно дифференцируема в точке $a$, если 

	$f$ дифференцируема в окрестности точки $a$ и 

	$d_x f$ непрерывна в точке $a$ ($\left\| d_x f - d_a f \right\| \to 0$ при $x \to a$)
\end{definition}

\begin{theorem}\slashns
	
	$f$ -- непрерывно дифференцируема в точке $a \iff$
	
	$f$ дифференцируема в окрестности точки $a$, все частные производные $f$ существуют в окрестности точки $a$ и непрерывны в точке $a$.

	(Большое спасибо Юре за исправление формулировки этой теоремы. У Ани тут забыто условие)
\end{theorem}


\begin{consequence}
	
	$f: E \to \R^m \;\; E \subset \R^n$
	$f$ -- непрерывно дифференцируема на $E \iff$ во всех точках из $E$ существуют все частные производные и они непрерывны.
\end{consequence}

\begin{proof} (Теоремы)\slashns
	
	
	``$\implies$''
	Разложим $f$ на координатные функции, рассмотрим одну из них - $f_k$. Продифференцируем её по какому-то $x_j$. Рассмотрим модуль разности значений в точках $x$ и $a$:

	$\left|\frac{\partial f_k}{\partial x_j}(x) - \frac{\partial f_k}{\partial x_j}(a)\right|$

	для доказательства оценим эту разность сверху чем-то стремящимся к 0.

	Так как $\frac{\partial f_k}{\partial x_j} (x) = \left<d_x f(e_j), e_k\right>$, то

	$\left|\frac{\partial f_k}{\partial x_j}(x) - \frac{\partial f_k}{\partial x_j}(a)\right| = \left|\left<d_x f(e_j)-d_a f(e_j), e_k\right>\right| \le \left\|d_x f(e_j)-d_a f(e_j)\right\| \left\|e_k\right\|=$

	$=\left\|d_x f(e_j)-d_a f(e_j)\right\|\le \left\|d_x f-d_a f\right\| \left\|e_j\right\|=\left\|d_x f-d_a f\right\|\to 0$
	
	``$\Longleftarrow$''

	Рассмотрим $\left\|d_x f - d_a f\right\|^2$

	Воспользуемся ранее доказаной теоремой, что квадрат нормы матрицы не превосходит суммы квадратов её коэффициентов:
	
	$\left\|d_x f - d_a f\right\|^2 \le \sum\limits_{k = 1}^{m}\sum\limits_{j = 1}^{n}(\frac{\partial f_j}{\partial x_k}(x) - \frac{\partial f_j}{\partial x_k}(a))^2 \to 0$, так как

	из $x\to a$ и непрерывности частных производных следует, что каждое слагаемое стремится к 0, их конечное кол-во, значит, и сумма стремится к 0, поэтому $\left\|d_x f - d_a f\right\|^2\to0$, а тогда и $\left\|d_x f - d_a f\right\|\to0$
\end{proof}

\begin{theorem}\slashns
	
	Непрерывная дифф. сохраняется при взятии линейной комбинации, композиции и скалярного произведения.
\end{theorem}


\begin{proof} \slashns
	
	Все эти действия сохраняют дифференцированность, нужно понять, что непрерывность у дифференциала тоже сохранится. По доказаной выше теореме для этого достаточно проверить непрерывность частных проивзодных.

	Докажем, например, композицию: Матрица производной композиции - произведение двух матриц с непрерывными коэффициентами, поэтому при произведении тоже получим непрерывные коэффициенты.

	Аналогично доказываются другие действия. Но Храбров не стал. Вообще в билетах про эту теорему ни слова, потому что нормально мы её не доказывали. Наверное, рассказывать её не стоит.
\end{proof}