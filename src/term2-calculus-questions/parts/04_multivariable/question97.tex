\newpage
\Subsection{Билет 97: Расстояние от точки до гиперплоскости.}

Это пример на использование метода множителей Лагранжа

\begin{example} \thmslashn
    
    $\R^n ~~ \langle a, x \rangle + b = 0 ~~~ a_1x_1 + \dotsc + a_nx_n + b = 0 ~~ $ гиперплоскость $L$
    
    $c$ -- точка $~~ \rho(c, L) := \inf\limits_{x \in L} ||x - c||$
    
    $f(x) = ||x - c||^2 = \sum\limits_{i = 1}^n (x_i - c_i)^2$ -- это функция, которую мы минимизируем.
    
    $\Phi(x) = \sum\limits_{i = 1}^n a_ix_i + b$
    
    Мы хотим минимизировать $f$ при условии, что $\Phi = 0$
    
    Что нам говорит метод множителей Лагранжа: надо завести вспомогательную функцию  $F = f - \lambda\Phi$, и для этой функции все частные производные должны равнятся нулю в точке условного экстремума: \large $\frac{\partial F}{\partial x_k}$ \normalsize $ = 0$
    
    \large $\frac{\partial F}{\partial x_k}$ \normalsize $= 2(x_k - c_k) - \lambda a_k = 0 ~~$ умножим на $a_k$ и сложим
    
    $2\langle x - c, a \rangle - \lambda \langle a, a \rangle = 0$
    
    $2\langle x - c, a \rangle = 2\langle x, a \rangle - 2\langle c, a \rangle = -2b - 2\langle c, a \rangle$
    
    $\lambda = $\large $\frac{-2\langle a, c \rangle - 2b}{||a||^2}$ \normalsize
    
    $x_k = c_k + \frac{\lambda}{2}a_k = c_k - $\large $\frac{\langle a, c \rangle + b}{||a||^2}$\normalsize$a_k$
    
    Найдём $f$ в этой $(\cdot)$: 
    
    $f(x) = \sum\limits_{k = 1}^n (x_k - c_k)^2 = \sum\limits_{k = 1}^n \left(\frac{\langle a, c \rangle + b}{||a||^2}\right)^2a_k^2 = $ \large $\frac{(\langle a, c \rangle + b)^2}{||a||^2}$ \normalsize \textcolor{red}{$ := R^2$}
    
    Значит, мы нашли единственную точку, в которой может быть экстремум, и нашли значение функции в этой точке. Если мы теперь докажем, что минимум обязан существовать, то это будет значить, что мы нашли именно минимум.
    
    Почему есть минимум: доказывается это с помощью теоремы Вейерштрасса, хотя тут никакой компактности нет.
    
    Есть непрерывная функция $f(x)$. Гиперплоскость -- не компакт. Проблема решается так: возьмём наименьшее растояние, которое у нас тут получилось, и назовём его $R^2$. Рассмотрим шарик радиуса $2R$. Тогда пересечение этого шарика с гиперплоскостью -- компактное множество (пересечение замкнутых множеств замкнуто, а шарик ограничен -- пересечение ограничено). 
    
    Часть гиперплоскости внутри шара $\overline{B}_{2R}(c)$ -- компакт. Функция непрерывна $\implies$ на этом компакте $f$ достигает минимум. Это и будет искомый минимум, потому что если возьмём точку где-то вне этого шара, то расстояние будет больше $R$.
    
    Ответ: $\rho(c, L) =$ \large $\frac{\langle a, c \rangle + b}{||a||}$ \normalsize
    
    \begin{remark} \thmslashn
    	
    	Когда мы решали систему уравнений, у нас были неизвестные: $x_k$ ($n$ штук) и $\lambda$. Итого $n + 1$ неизвестных. Как получилось решить? На самом деле у нас есть ещё одно уравнение, которым мы тоже пользовались, но о котором умолчали. Мы ищем точку на поверхности, поэтому $\Phi = 0$.
    	
    \end{remark}
    
\end{example} 

