\Subsection{Билет 74: Дифференцируемость координатных функций. Примеры дифференцируемых отображений}

Разберем несколько примеров дифференцируемых отображений.

\begin{example} \thmslashn

    $f(x) = const = c$ 

    $f(a + h) = f(a)$

    $T \equiv 0$, $\alpha \equiv 0$

\end{example}

\begin{example} \thmslashn

    $f$ - линейное отображение

    $f(a + h) = f(a) + f(h) = f(a) + T(h)$

    $T = f$, $\alpha \equiv 0$ 

\end{example}

\begin{definition} \thmslashn

    $f: E \mapsto \mathbb{R}^m, \ E \subset \mathbb{R}^n$

    $
        f(x) = 
        \begin{pmatrix}
            f_1(x)\\
            f_2(x)\\
            \vdots\\
            f_m(x)
        \end{pmatrix}
     f_k:  E \subset \mathbb{R}^n \mapsto \mathbb{R}$ - координатные функции


\end{definition}

\begin{theorem} \thmslashn

    Дифференцируемость функции $f$ в точке $a$ равносильна дифференцируемости в точке $a$

    всех ее координатных функций.

    \begin{proof} \thmslashn
        
        $f: E \mapsto R^m, E \subset \mathbb{R}^n$

        \begin{enumerate}
            \item $\Rightarrow$
                $f(a + h) = f(a) + Th + \alpha(h)$, где $\alpha(h) = o(||h||)$

                Распишем определение в виде векторного равенства:

                $
                \begin{pmatrix}
                    f_1(a + h)\\
                    f_2(a + h)\\
                    \vdots\\
                    f_m(a + h)
                \end{pmatrix} = 
                \begin{pmatrix}
                    f_1(a)\\
                    f_2(a)\\
                    \vdots\\
                    f_m(a)
                \end{pmatrix} +
                \begin{pmatrix}
                    T_1(h)\\
                    T_2(h)\\
                    \vdots\\
                    T_m(h)
                \end{pmatrix} +
                \begin{pmatrix}
                    \alpha_1(h)\\
                    \alpha_2(h)\\
                    \vdots\\
                    \alpha_m(h)
                \end{pmatrix}  
                $
                
                Рассмотрим конкретную координату:    

                $f_k(a + h) = f_k(a) + T_k h + \alpha_k(h)$ 

                $T_k h$ - произведение строки матрицы на вектор, поэтому - линейное отображение

                (по сути просто скалярное произведение).

                Необходимо показать, что $\dfrac{\alpha_k(h)}{||h||} \to 0$ при $h \to 0$

                Достаточно заметить, что так как $||\alpha(h)|| = \sqrt{\sum\alpha_k(h)^2}$, то $|\alpha_k(h)| \le ||\alpha(h)||$

                Но тогда $\dfrac{|\alpha_k(h)|}{||h||} \le \dfrac{||\alpha(h)||}{||h||} \to 0$

                Отсюда получаем вывод, что $\dfrac{\alpha_k(h)}{||h||} \to 0$, получается,

                что доказали для всех координатных функций.
            \item $\Leftarrow$

                Знаем, что $f_k(a + h) = f_k(a) + T_k h + \alpha_k(h)$ и $\dfrac{\alpha_k(h)}{||h||} \to 0$ при $h \to 0$

                Соберем все это в один вектор, из строчек $T$ получаем матрицу, тогда в результате:

                $f(a + h) = f(a) + T h + \alpha(h)$

                Надо проверить, что $\dfrac{\alpha(h)}{||h||} \to 0$, т. е. $\dfrac{||\alpha(h)||}{||h||} \to 0$ 

                $\dfrac{||\alpha(h)||}{||h||} = \dfrac{\sqrt{\alpha_1(h)^2 + \ldots + \alpha_m(h)^2 }}{||h||} = 
                \sqrt{\dfrac{\alpha_1(h)^2}{||h||^2} + \ldots + \dfrac{\alpha_m(h)^2}{||h||^2}} \to 0$
        \end{enumerate}

    \end{proof}

\end{theorem}

\begin{consequence} \thmslashn

    Строки матрицы Якоби - градиенты координтных функций.

    \begin{proof} \thmslashn

        Строки матрицы Якоби - это те самые $T_k$, которые встречались в доказательстве теоремы. 

        Тогда можно заметить, что $T_k h = \langle T_k^T, h \rangle$

        Отсюда получаем, что строки матрицы Якоби - градиенты координтных функций.
    \end{proof}

\end{consequence}