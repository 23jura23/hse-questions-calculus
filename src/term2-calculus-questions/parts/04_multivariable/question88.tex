\Subsection{Билет 88: Оценка на норму разности значений дифференцируемого отображения. Оценка на норму обратного отображения. Теорема об обратимости отображений, близких к обратимым}

\begin{theorem}[Оценка на норму обратного отображения] \thmslashn

	Если $A : R^n \rightarrow R^n$ линейное, $\|Ax \| \geq m\|x\|$ $\forall x \in R^n$ и $m>0$, тогда $A$ - обратим и $\|A^{-1}\| \leq \frac{1}{m}$
	\begin{proof} \thmslashn

		Нужно проверить иньективность (точки не склеиваются). Так как $A$ линейно, нужно проверить, что $A$ ничего не переводит в ноль, кроме нуля. То есть доказать, что $Ax = 0 \iff x = 0$.
		
		Если $Ax = 0$, то $\|Ax\| = 0 \geq m\|x|\ \Rightarrow x = 0$.
		
		Раз точки не склеиваются, значит $\exists A^{-1}$. Осталось оценить ее норму. Пусть $y = A^{-1}x$, тогда...
		
		$$
		\|A^{-1}\|
		=
		\sup_{x\neq0}\frac{\|A^{-1}x\|}{\|x\|}
		=
		\sup_{y\neq0}\frac{\|y\|}{\|Ay\|}
		\leq
		\sup_{y\neq0}\frac{\|y\|}{m\|y\|}
		=
		\frac{1}{m}
		$$
		Что и требовалось доказать.
	\end{proof}
\end{theorem}

\begin{theorem}[Оценка на норму разности значений дифференцируемого отображения] \thmslashn

	$f: R^n \rightarrow R^m$ дифференцируема в $B_r(a)$ и $\|f^\prime(x)\| \leq \alpha$ \; $\forall x \in B_r(a)$, тогда $\|f(x) - f(y)\| \leq \alpha\|x - y\|$
	\begin{proof} \thmslashn

		$\phi(t) :=  \langle f(x + t(y-x)), f(y) - f(x) \rangle$.
		
		Воспльзуемся линейностью скалярного произведения. Далее применим формулу Лагранжа и возьмем $\xi \in (0, 1)$.
		
		$$
		\|f(y) - f(x)\|^2
		=
		\phi(1) - \phi(0)
		=
		\phi^\prime(\xi)
		$$
		
		$$
		\phi^\prime(\xi) = \langle...\rangle^\prime 
		= 
		\langle(f(x + t(y - x)))^\prime, f(y) - f(x)\rangle
		=
		\langle f^\prime (x + t(y - x))(x + t(y - x))_t^\prime, f(y) - f(x)\rangle
		=
		$$
		$$
		=
		\langle f^\prime(x + t(y-x))(y - x), f(y) - f(x)\rangle
		$$
		
		Подставим функцию. Оценим скалярное произведение. Замечание: точка $(x + \xi(y - x))$ находится между $x$ и $y$, а значит живет в шаре $B_r(a)$. Тогда $f^\prime(x + \xi(y - x))$ - это произведение матрицы на вектор.
		
		$$
		\phi^\prime(\xi)
		=
		\langle f^\prime(x + \xi(y - x), f(y) - f(x)) \rangle
		\leq
		\|f^\prime(x + \xi(y - x))\|\|f(y) - f(x)\|
		\leq
		\alpha\|y-x\|\|f(y) - f(x)\|
		$$
		
		Вспомним, откуда мы начинали.
		
		$$
		\|f(y) - f(x)\|^2 
		=
		\phi^\prime(\xi)
		\leq 
		\alpha\|y-x\|\|f(y) - f(x)\|
		$$
		
		Тогда можно сократить $\|f(y) - f(x)\|$ и теорема будет доказана.
	\end{proof}
\end{theorem}

\begin{theorem}[Об обратимости оператора близкого к обратимому] \thmslashn
	
	$A: R^n \rightarrow R^n$ обратим и $\|B - A\| < \frac{1}{\|A^{-1}\|}$. Тогда $B$ - обратим, $\|B^{-1}\| \leq \frac{1}{\|A^{-1}\| - \|B - A\|}$ и $\|B^{-1} - A^{-1}\| \leq \frac{\|A^{-1}\| \|B - A\|}{\|A^{-1}\|^{-1} - \|B - A\| }$
	\begin{proof} \thmslashn
		
		 Воспользуемся неравенством треугольника.
		
		$$
		\|Bx\|
		\geq
		\|Ax\| - \|(B - A)x\|
		\geq
		\frac{\|x\|}{\|A^{-1}\|} - \|B - A\|\|x\|
		=
		\|x\|(\frac{1}{\|A^{-1}\|} - \|B - A\|)
		$$
		
		Откуда взялся предпоследний переход? Заметим, что $\|(B - A)x\| \leq \|B  - A\| \|x\|$. Так же подметим, что 
		$$
		\|A^{-1}\| \|Ax\| \geq \|A^{-1}Ax\|
		=
		\|x\| 
		\iff
		\|Ax\|
		\geq 
		\frac{\|x\|}{\|A^{-1}\|}$$
		
		Пусть $m:= (\frac{1}{\|A^{-1}\|} - \|B - A\|)$. Тогда $ \|Bx\| \geq m\|x\|$, $\forall x \in R^n \Rightarrow B$ - обратима и $B^{-1} \leq \frac{1}{m}$ по предыдущим теоремам. 
		
		Воспользуемся линейностью.
		
		$$
		B^{-1} 
		=
		B^{-1}(AA^{-1} - BA^{-1})
		=
		B^{-1}(A - B)A^{-1}.
		$$
		
		$$
		\|B^{-1} - A^{-1}\| 
		=
		\|B^{-1}(A-B)A^{-1}\|
		\leq
		\|B^{-1}\|\|A - B\|\|A^{-1}\|
		\leq
		\frac{\|A - B\|\|A^{-1}\|}{m}
		$$
		
		Что и требовалось доказать.
	\end{proof}
\end{theorem}

\begin{remark}[(Нет названия)] \thmslashn
	
	Замечание: самое главное в этой формуле то, что $\|B - A\|$ находится в числителе. Это означает, что при $B \rightarrow A$ последовательность обратных будет стремится к обратным. Остальное в этой формуле маловажно.
\end{remark}
