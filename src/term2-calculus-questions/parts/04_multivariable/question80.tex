\Subsection{Билет 80: ! Связь частных производных и дифференцируемости.}

\begin{theorem} \thmslashn

    $f : E \in \R^n \mapsto \R, a \in Int E$.

    В окрестности точки a существуют все частные производные и они непрерывны в точке $a$.

    Тогда $f$ дифференцируема в точке $a$.


    \begin{proof} \thmslashn

	По сути, мы знаем, как должно быть устроено линейное отображение из определения дифференцируемости, т.к. нам известны частные производные (подробнее об этом расписано в билете 76)

	$R(h) := f(a + h) - f(a) - \sum_{k = 1}^n f'_{x_k}(a)h_k$

	Надо доказать, что $\frac{R(h)}{\left\| h \right\|} \to 0$ при $h \to 0$

	Заведем вспомогательные вектора: $b_k = (a_1 + h_1, ..., a_k + h_k, a_{k + 1}, ..., a_n)$, заметим, что тогда получается $b_0 = a, b_n = a + h$

	Рассмотрим вспомогательные функции одной переменной $F_k(t) := f(b_{k - 1} + th_ke_k)$, здесь $e_k$ - это стандартный вектор
 
	Запишем в координатном виде: $F_k(t) := f(a_1 + h_1, ..., a_{k - 1} + h_{k - 1}, a_k + th_k, a_{k + 1}, ..., a_n)$

	Применим одномерную теорему Лагранжа: $\underbrace{F_k(1) - F_k(0)}_{f(b_k) - f(b_{k - 1})} = F_k'(\Theta_k) = h_kf'_{x_k}(a_1 + h_1, ..., a_{k - 1} + h_{k - 1}, a_k + \Theta_kh_k, a_{k + 1},...,a_n) = h_kf'_{x_k}(c_k)$ для некоторой $\Theta_k \in (0, 1)$

	Получили, что $f(b_k) - f(b_{k - 1}) = h_kf'_{x_k}(c_k)$. Сложим все получившиеся равенства: $f(b_n) - f(b_0) = f(a + h) - f(a) = \sum_{k = 1}^n h_kf'_{x_k}(c_k) = \sum_{k = 1}^n h_kf'_{x_k}(a) + \sum_{k = 1}^n h_k(f'_{x_k}(c_k) - f'_{x_k}(a))$

	Заметим, что $\sum_{k = 1}^n h_k(f'_{x_k}(c_k) - f'_{x_k}(a))$ - формула для остатка $R(h)$

	$\left| R(h)\right| \le \left\| h \right\| (\sum_{k = 1}^n(f'_{x_k}(c_k) - f'_{x_k}(a))^2)^{\frac{1}{2}}$(КБШ)

	$\iff \frac{R(h)}{\left\| h \right\|} \le (\sum_{k = 1}^n(f'_{x_k}(c_k) - f'_{x_k}(a))^2)^{\frac{1}{2}} \to 0$ при $h \to 0$ по непрерывности частных производных

    \end{proof}

    \item Замечание 1. В формулировке теоремы интересуемся дифференцируемостью скалярной функции, но дифференцируемость векторнозначной функции равносильна дифференцируемости каждой ее координатной функции, которая есть скалярная функция.

    \item Замечание 2. Можно не требовать непрерывность ровно одной из частных производных
	\begin{proof} \thmslashn

	Не требуем непрерывность $f'_{x_1}$ в точке а. Необходимо, чтобы $f'_{x_1}(c_1) - f'_{x_1}(a) \to 0$.

	Нас интересует разность $f(b_1) - f(b_0) = f(a_1 + h, a_2, ..., a_n) - f(a_1, ..., a_n)$. То есть получили функцию, у которой последние координаты зафиксированы, а первую меняем. Такая функция дифференцируема в точке $a_1$ по определению частной производной.

	$f(a_1 + h, a_2, ..., a_n) = f(a_1, ..., a_n) + f_{x_1}'(a_1, ..., a_n)h_1 + o(h_1)$

	$f(b_1) - f(b_0) = f_{x_1}'(a)h_1 + o(h_1)$
	\end{proof}
    \item Замечание 3. Дифференцируемость в точке не дает существование част.производных в окрестности и тем более их непрерывность
	\begin{example} \thmslashn

	$f(x, y) = x^2 + y^2$, если ровно одно из чисел x или y рационально

	$f(x, y) = 0$ иначе 

	f непрерывна только в точке (0, 0), в остальных точках нет непрерывности ни по какому направлению

	Проверим дифференцируемость в (0, 0): $f(h, k) = \underbrace{f(0, 0)}_{=0} + Ah + Bk + o(\sqrt{h^2 + k^2})$

	$f(h, k) = o(\sqrt{h^2 + k^2})$, верно, т.к. $0 \le f(h, k) \le h^2 + k^2$
	\end{example}

\end{theorem}

