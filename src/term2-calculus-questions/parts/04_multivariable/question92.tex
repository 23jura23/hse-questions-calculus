\Subsection{Билет 92: Задача Коши для дифференциального уравнения. Теорема Пикара}

\begin{definition} \thmslashn
  
  Задачей Коши называется задача нахождения функции $y(x)$, удовлетворяющей следующим условиям:
  \[\begin{aligned}
    \begin{cases}
      \frac{dy}{dx} = f(x, y(x)) \\
      y(x_0) = y_0
    \end{cases}
  \end{aligned}\]
  Первое условие значит, что если продиффренировать функцию $y(x)$, то получим выражение, которе зависит от $x$ и от $y(x)$. Например $y = e^{x^2} \implies \frac{dy}{dx} = 2xe^{x^2} = 2xy(x)$
  Второе условие нужно, так как функций, подходящих под первое условие может быть много, поэтому можно ограничить таким образом.
\end{definition}

\begin{theorem}[Пикара] \thmslashn

  $D \subset \R^2$ - открытое. Если $f : D \mapsto \R$ непрерывна, $(x_0, y_0) \in D$ и $|f(x, y) - f(x, \tilde y)| \le M|y - \tilde y| \quad \forall (x, y)$ и $(x, \tilde y)$,
  то при некотором $\delta > 0$ на отрезке $[x_0 - \delta, x_0 + \delta]$ существует единственная функция $\phi$, являющаяся решением задачи Коши, то есть
  \[\begin{aligned}
    \begin{cases}
      \phi(x_0) = y_0\\
      \phi'(x) = f(x, \phi(x))
    \end{cases}\text{ для } x \in [x_0 - \delta, x_0 + \delta]
  \end{aligned}\]
  
  \begin{remark} \thmslashn

    Почему $f : \textbf{D} \mapsto \R$, то есть почему бы не писать $f : \R^2 \mapsto \R$? Потому что существуют такие $f$, что решение на отрезке есть, а на всей прямой нет.
    \begin{example} \thmslashn

      Задача Коши:
      \[\begin{cases}
        y' = -y^2 \\
        y(1) = 1
      \end{cases}\]
      Как бы мы ни старались подобрать $x_0$ и $y_0$ у нас не получится получить решение, такое чтобы оно включало точку 0. То есть в данной теореме важна локальность.
    \end{example}
  \end{remark}

  \begin{proof} \thmslashn

    Перейдем от системы к немного другому уравнению, а именно: 
    \[
      \phi(x) = y_0 + \int\limits_{x_0}^{x}{f(t, \phi(t))\,dt}
    \]
    Действительно, если мы докажем существование такой $\phi(x)$, то мы решим задачу Коши, так как
    \[\begin{cases}
      \phi(x_0) = y_0 + \int\limits_{x_0}^{x_0}{f(t, \phi(t))\,dt} = y_0 \\
      \phi'(x) = 0 + \left(\int\limits_{x_0}^{x}{f(t, \phi(t))\,dt}\right)' = f(x, \phi(x))
    \end{cases}\]
    \par Выберем такое $r \in \R$, что $B_r(x_0, y_0) \subset \overline{B}_r(x_0, y_0) \subset D$. Так можно выбрать, так как $D$ - открытое.
    Так как $\overline{B}_r(x_0, y_0)$ - компакт, и $f$ - непрерывна, то $f$ - ограничена на $\overline{B}_r(x_0, y_0)$. Пусть $|f(x, y)| \le K$ на $\overline{B}_r(x_0, y_0)$.
    \par Теперь выберем $\delta$. Оно должно соответствовать двум условиям. 
    \begin{enumerate}
      \item хочется, чтобы прямоугольник $[-\delta, \delta] \times [-K\delta, K\delta]$ с центром (точка пересечения диагоналей) в $(x_0, y_0)$ полностью лежал внутри $B_r(x_0, y_0)$. Более формально: 
    \[ \text{Если } |x - x_0| < \delta \text{ и } |y - y_0| < K\delta \text{, то } (x, y) \in B_r(x_0, y_0) \]
      \item хочется, чтобы $M\delta < 1$, $M$ - из условия теоремы.
    \end{enumerate}
    Оба эти условия несложно удовлетворить.
    \par $C^* := \{\phi \in C[x_0 - \delta, x_0 + \delta] : |\phi(x) - y_0| \le  K\delta\}$. В данном пространстве зададим стандартную для непрерывных функций метрику - максимум модуля разности.
    Докажем, что данное пространство полное. Данне пространство является подпространсвтом полного, надо доказать, что такое пространство - замкнуто. Оно замкнуто, так как если есть последовательность функций из $C^*$, то и их предел, будет лежать в $C^*$, так как в переделе нестрогое неравенство сохраняется. 
    \par В данном пространстве возьмем отображение $T(\phi) = \psi$, где $\psi(x) = y_0 + \int\limits_{x_0}^{x} f(t, \phi(t))\, dt$
    Докажем, что $T:C^* \mapsto C^*$ и $T$ - сжатие. Если это доказать, то по Теореме Бонаха $T$ будет имееть единственную неподвижную точку, значит $\exists \phi(x) = y_0 + \int\limits_{x_0}^{x} f(t, \phi(t))\, dt$.
    \\ При действии такого отображения на непрерывную функцию получится также непрерывная функция. Теперь докажем, что
      \[ \text{Если } |\phi(x) - y_0| \le K\delta \implies |\psi(x) - y_0| \le K\delta \]
    $|\psi(x) - y_0| = \left|\int\limits_{x_0}^{x}{f(t, \phi(t))\, dt}\right| \le \{\text{максимум функции}\} \cdot \{\text{длина отрезка}\} \le K\delta$ \\
    $f(t, \phi(t)) \le K$, так как $(t, \phi(t)) \in B_r(x_0, y_0)$.
    \\ Теперь проверим, что $T$ - сжатие. $T(\phi) = \psi, T(\tilde\phi) = \tilde\psi$
    \[\begin{aligned} |\psi(x) - \tilde\psi(x)| = \left|\int\limits_{x_0}^{x}f(t, \phi(t))\, dt - \int\limits_{x_0}^{x}f(t, \tilde\phi(x))\, dt\right| \le \int\limits_{x_0}^{x}|f(t, \phi(t)) - f(t, \tilde\phi(x))|\, dt \end{aligned}\]
    Вспомним, что в условии сказано: $|f(x, y) - f(x, \tilde y)| \le M|y - \tilde y|$. Значит можно продолжить цепочку неравенств. 
    \[ |\psi(x) - \tilde\psi(x)| \le \int\limits_{x_0}^{x}M|\phi(t) - \tilde\phi(t)|\, dt \le M\delta\max\{|\phi(x) - \tilde\phi(x)|\} = M\delta||\phi - \tilde\phi|| \]
    С левой стороны неравенства также рассмотрим максимум разностей и получим, что
    \[ ||\psi(x) - \tilde\psi(x)|| \le \underbrace{M\delta}_{cosnt < 1} ||\phi(x) - \tilde\phi(x)|| \]
    Получается $T$ - сжатие по определению, значит существует единственная $\phi(x)$, удовлетворяющая условиям задачи Коши на отрезке $[x_0 - \delta, x_0 + \delta]$.
  \end{proof}
\end{theorem}

