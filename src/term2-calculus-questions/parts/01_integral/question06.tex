\Subsection{Билет 6: Формула Стирлинга}
Продолжаем примеры для формулы Эйлера-Маклорена
\begin{example} \thmslashn
\begin{enumerate}
    \item[3.] \textbf{Формула Стирлинга}
    
    Хотим найти $\ln (n!)$
    
    Пусть $f(t) = \ln t$, тогда $f''(t) = -\frac{1}{t^2}$
    
    $\sum\limits_{k=1}^{n} \ln k = \ln (n!) = \int\limits_{1}^{n} \ln t \,dt + \frac{\ln 1 + \ln n}{2} + \frac{1}{2} \int\limits_{1}^{n} -\frac{1}{t^2} \cdot \{t\}(1 - \{t\})\,dt$
    
    $b_n := \int\limits_{1}^{n} \frac{\{t\}(1 - \{t\})}{t^2}\,dt \leqslant b_{n+1} \Rightarrow$ $b_n$ возрастает.
    
    $b_n = \int\limits_{1}^{n} \frac{\{t\}(1 - \{t\})}{t^2}\,dt \leqslant \frac{1}{4}\int\limits_{1}^{n} \frac{dt}{t^2} = \left.\frac{1}{4} \left( -\frac{1}{t} \right) \right|_{t=1}^{t=n} = \frac{1}{4} - \frac{1}{4n} \leqslant \frac{1}{4} \Rightarrow$ $b_n$ сходятся. $\Rightarrow b:= \lim b_n \Rightarrow b_n = b + o(1)$
    
    $\ln(n!) = n\ln n + \frac{\ln n}{2} -n -\frac{b}{2} + o(1)$
    
    $n! = n^n e^{-n} \sqrt{n} e^{-\frac{b}{2}} e^{o(1)} \sim n^ne^{-n}\sqrt{n}e^{1-b}$, т.к. $e^{o(1)} \to 1$
    
    Хотим понять, что такое $c := e^{1-b}$.
    
    $\binom{2n}{n} \sim \frac{4^n}{\sqrt{\pi n}}$
    
    $\binom{2n}{n} = \frac{(2n)!}{(n!)^2} \sim \frac{(2n)^{2n} e^{-2n}\sqrt{2n} c}{n^{2n} e^{-2n} n c^2} = \frac{2^{2n} \sqrt{2n}}{n\cdot c} = \frac{4^n \sqrt{2}}{\sqrt{n} \cdot c} \Rightarrow c = \sqrt{2 \pi}$
    
    
    Итого формула Стирлинга:
    
    $n! \sim n^ne^{-n}\sqrt{2 \pi n}$
    
\end{enumerate}
\end{example}