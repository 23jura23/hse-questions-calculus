\Subsection{Билет 11:  Признак Абеля. Интеграл от произведения монотонной и периодической функций. Интеграл $\int\limits_1^{+\infty} \frac{\sin x}{x^p}\, dx$}

\newcommand*{\abs}[1]{\left|\,{#1}\,\right|} % вынести куда-то?

\begin{theorem}[признак Абеля]\slashns
	
	$f,g \in C[a,+\infty)$
	
	\begin{enumerate}
		\item $\int\limits_a^{+\infty} f(x) \, dx$ -- сходится
        \item $\abs{g(x)}$ - ограничена, то есть $\abs{g(x)} \le K \;\; \forall x > a$ 
		\item $g$ монотонна
	\end{enumerate}
	
	Из этого всего следует, что $\int\limits_a^{+\infty} fg$ сходится
\end{theorem}

\begin{proof}\slashns
	
	Будем доказывать через Дирихле, то есть заделаем себе такие функции из $f, g$, что они будут сходится по признаку Дирихле.

    Напомним его условия: у одной из функций интеграл должен быть ограничен, другая монотонно стремится к 0. С помощью $f(x)$ мы получим ограниченный интеграл, с помощью $g$ - монотонно стремяющуюся к 0 функцию.
	
    Интеграл $\int\limits_a^{+\infty} f(x)\,dx$ сходится, то есть по определению получаем, что существует конечный предел: $\exists \lim\limits_{y \to +\infty} \int\limits_a^y f(x)\, dx$. 

    Нам в определении нужно, чтобы $\exists C: \forall b: \abs{\int\limits_{a}^{b} f(x)} < C$, то есть ограниченность интеграла. Выберем какой-нибудь отрезок $[a, B]$.

    \begin{enumerate}
        \item $b \le B$. То есть мы хотим ограничить интеграл на отрезке. Он на нём непрерывен (вроде очев, но допустим, потому что дифференцируем :), так что по т. Вейерштрасса, ограничен, что и хотели.
        \item $b > B$. То есть $b$ где-то в окрестности бесконечности. Но у нас есть предел интеграла на бесконечности - он достигается и конечный. Так что можно ограничить $\abs{\int\limits_a^b f(x)}$ через предел + константу при достаточно больших $b$. 

            Отсюда вроде получаем, что $B$ надо брать достаточно большое, то есть чтобы можно было такую константу вообще найти.
    \end{enumerate}

    (у Храброва и Ани так подробно про выбор отрезка не было, там просто говорилось, то на маленьких непрерывность даёт ограниченность, а на бесконечности предел, я попробовал раскрыть. Возможно, что так подробно не стоит рассказывать на экзе)
	
    Далее:
    $g$ монотонна и ограничена $\implies \exists A:= \lim\limits_{x \to +\infty} g(x)$ и $\abs{A} \le K$. То есть возьмём и найдём её предел, он конечный.
	
	$\widetilde{g}(x) := g(x) - A$ монотонна и стремится к $0$ на бесконечности.
	
	Т.е. показали, что $f$ и $\widetilde{g}$ удовлетворяют условию принципа Дирихле.
	
	$\implies \int\limits_a^{+\infty} f(x) \widetilde{g}(x)\, dx$ -- сходится
	
	$fg = fA+f\widetilde{g}$. А в интеграле: $\int\limits_a^{+\infty} fg = \int\limits_a^{+\infty} fA + \int\limits_a^{+\infty} f\widetilde{g}$

    $\int\limits_a^{+\infty} fA$ сходится, так как $A$ - константа и её можно вынести, а $\int\limits_a^{+\infty} f$ сходится по условию признака.

    А второго слагаемое сходится по доказаному выше. Итого оба слагаемых сходятся, то есть и интеграл сходится, ура.
\end{proof}

\begin{consequence}[(интеграл от произведения монотонной и периодической функции)]\slashns
    
	$f, g \in C[a,+\infty)$ и $f$ периодична с периодом $T$, $g$ - монотонна.

    \begin{enumerate}
        \item $\int\limits_a^{+\infty} g(x)\, dx$ -- сходится. 

            Тогда $\int\limits_a^{+\infty}f(x) g(x)\, dx$ -- сходится абсолютно.
        
        \item $\int\limits_a^{+\infty} g(x)\, dx$ -- расходится. Дополнительное условие: $g(x) \to 0$ на бесконечности.

            Тогда $\int\limits_a^{+\infty}f(x) g(x)\, dx$ -- сходится $\iff \int\limits_a^{a+T} f(x)\, dx = 0$
            
            (кстати, в этом условии хватит и того, что $\int\limits_a^{+\infty} \abs{g(x)}\, dx$ расходится абсолютно: об этом в доказательстве)

    \end{enumerate}

    Это следствие из признака Дирихле (не Абеля)

\end{consequence}

\begin{proof}\slashns

    \begin{enumerate}
        \item Первый случай: интеграл $g$ сходится 

            $f \in C[a, a+T]$, то есть непрерывна на отрезке, то есть ограничена на нём по т. Вейерштрасса. При этом она периодическая, так что те же ограниченные значения будут и на всей прямой. То есть, она ограничена везде.

            $g$ монотонна, то есть только 1 раз может пересечь 0, а дальше она знакопостоянна. То есть на неком луче $[b, +\infty$ она знакопостоянна. Допустим, что она на нём положительна, отсюда $\abs{g} = g$, пригодится ниже. Если же она отрицательна, то можем рассматривать $-g$ в условии, интеграл просто поменяет знак, то есть сходимость это не влияет.

            Нам теперь интересно: $\int\limits_b^{+\infty} \abs{fg}$ - сходится? Но $f$ ограничена, то есть не больше какой-то константы $M$; а $\abs{g} = g$, так что:

            $\int\limits_b^{+\infty} \abs{fg} \le \int\limits_b^{+\infty} Mg = M\int\limits_b^{+\infty} g$

            По условию этого случая, интеграл $g$ сходится, то есть получили, то и исходный интеграл $\abs{fg}$ тоже сходится, ура.

            Небольшое пояснение: мы же доказали сходимость на луче $[b, +\infty]$? Да, а на полуинтервале $[a, b)$ интеграл просто конечен и на сходимость на $[a, +\infty]$ не влияет.

        \item Второй случай: интеграл $g$ абсолютно расходится, $g \to 0$. 

	``$\Longleftarrow$''
	
    $F(y) := \int\limits_a^y f(x) \, dx = \underbrace{\int\limits_{a}^{a+kT} f(x) \, dx}_{= 0} + \int\limits_{a+kT}^y f(x) \, dx = \int\limits_{a+kT-kT}^{y-kT} f(x) \, dx = \int\limits_{a}^{y-kT} f(x) \, dx$
	
	$a \le y - kT \le a+ T$.

    То есть мы хотим посчитать интеграл. Разобьём его на интеграл, который включает в себя целое число периодов: на каждом из таких периодов он 0 (интеграл по периоду 0, не забываем), так что и весь интеграл по целому числу периодов равен 0. Оставшийся интеграл - остаток от $a+kT$ до $y$. Так как функция у нас периодическая, то мы можем сдвинуть границы на $T$ и ничего не изменится, так что можно сдвинуть и на $kT$ влево, и получить интеграл от $a$ до $y-kT \in [a, a+T]$.  
	
    То есть получили, что множество значений $F(y)$ при $y \in \R$ и множество значений $F(y)$ при $y \in [a,a+T]$ совпадает: $\forall y \in \R$ мы найдём значение функции, которая считалась только на отрезке $[a,a+T]$.
	
    Но $F$ непрерывна $\implies$ ограничена на $[a,a+T]$ по т. Вейерштрасса $\implies F$ ограничена на $\R$, ведь множество значений совпадает, а значит максимальное значения функции и на $[a, a+T]$, и на $\R$ совпадают, то есть максимум ограничен и там, и там. (и минимум тоже ограничен, если функция очень маленькя и в минус уходит)
	
    $\implies$ по принципу Дирихле $\int\limits_a^{+\infty}f(x)g(x)\, dx$ сходится: $\int\limits_a^{+\infty} f(x)\, dx$ ограничен, а $g$ монотонна и стремится к $0$.
	
	``$\implies$''
	
    От противного.

    Докажем, что если $\int\limits_a^{a+T} f(x)\, dx =:A\ne 0$, то $\int\limits_a^{+\infty} fg$ расходится. (то есть если интеграл по периоду не 0, то расходится)
	
    $\tilde{f}(x):= f(x) - \frac{A}{T}$ -- периодическая, так как просто отняли константу ($A,T$ - константы).
	
    $\int\limits_a^{a+T} \tilde{f} = \int\limits_a^{a+T} f - \int\limits_a^{a+T} \frac{A}{T} = A - A = 0$. (просто $f(x) = \tilde{f}(x) + \frac{A}{T}$ и мы разбили интеграл) 
	
    Значит, $\int\limits_a^{+\infty} \tilde{f}g$ сходится по следствию $\Longleftarrow$, так как интеграл $\tilde{f}$ равен 0 по периоду, а $g$ монотонно стремится к 0. 
	
	Но $\int\limits_a^{+\infty} fg = \int\limits_a^{+\infty} \tilde{f}g + A\int\limits_a^{+\infty}g$

    Первое слагаемое сходится, как только что доказали. Второе слагаемое расходится - по условию, $\int\limits_a^{+\infty} g$ расходится. То есть сходящийся интеграл = сходящийся + расходящийся, что невозожно, противоречие, что и нужно было.

    Теперь о том, почему хватит расходимости $\int\limits_a^{+\infty} \abs{g}$, а не  $\int\limits_a^{+\infty} g$, как в условии.

    Ну вообще, из расходимости интеграла следует и абсолютная расходимость: интеграл абсолютно сходится $\implies$ интеграл сходится, и не может быть так, что интеграл не сходится просто, но сходится абсолютно

    Но на самом деле, из абсолютной расходимости в данной задаче следует и обычная расходимость, потому что $g$ монотонна. Она лишь 1 раз может пересечь 0 и поменять знак. И после пересечения 0 она будет знакопостоянна. То есть интеграл на хвосте будет совпадать с абсолютным интегралом в точности до знака. А значит, у них будет одновременная сходимость и расходимость

    То есть если бы мы в условии написали абсолютную расходимость, то из неё бы мы получили и обычную расходимость, для которой мы теорему доказали.

    \end{enumerate}
	
\end{proof}

\begin{example}\slashns
	
	$\int\limits_1^{+\infty} \frac{\sin x}{x^p} \, dx$
	
	Случай 1.
	
	$p  > 1 \;\; \frac{\abs{\sin x}}{x^p} \le \frac{1}{x^p} \;\; \int\limits_1^{+\infty} \frac{dx}{x^p}$ сходится
	
	$\implies \int\limits_a^{+\infty} \frac{\sin x}{x^p}\, dx$ -- абсолютно сходится.
	
	Случай 2.
	
	$0 < p \le 1$
	
	$\sin x$ -- периодическая функция $\int\limits_0^{2\pi} \sin x\, dx = 0$
	
	$\frac{1}{x^p} \to 0$ и монотонна. 
	
    $\implies $ по следствию из признака Дирихле $\int\limits_1^{+\infty} \frac{\sin x}{x^p} \, dx$ сходится.
	
	Покажем, что в этом случае нет абсолютной сходимости.
	
	$\abs{\sin x}$ -- периодическая функция, $\int\limits_0^{2\pi} \abs{\sin(x)}\, dx \ne 0$, так что по тому же следствию, получаем, что интеграл расходится.
	
	$\implies$ нет абсолютной сходимости.
	
	Случай 3.
	
	$p \le 0$
	
    Воспользуемся критерием Коши.
	
	$\int\limits_{\pi/6+2\pi k}^{5\pi/6 + 2\pi k} \frac{\sin x}{x^p} \,dx \ge \int\limits_{\pi/6+2\pi k}^{5\pi/6 + 2\pi k} \frac{1/2}{x^p} \,dx \ge \int\limits_{\pi/6+2\pi k}^{5\pi/6 + 2\pi k} \frac{1}{2} \,dx = \frac{\pi}{3}$
	
	$\implies$ нет сходимости, так как не стремится к 0.
\end{example}


