\Subsection{Билет 3: Эквивалентная для суммы $\sum\limits_{k = 1}^n k^p$. Формула трапеций.}

\begin{example}\slashns

	$S_n(p) = 1^p + 2^p + 3^p +...+n^p$
	
	Ограничим $S_n(p)$ сверху: $S_n(p) < n \cdot n^p = n^{p+1}$
	
	Чтобы ограничить снизу, возьмем только вторую половину слагаемых. Заметим, что каждое слагаемое $\geq\frac{n}{2}$. Получаем: $S_n(p) > \frac{n}{2} (\frac{n}{2})^p = \frac{n^{1+p}}{2^{1+p}}$
	
	$\frac{n^{1+p}}{2^{1+p}}  <S_n(p) < n^{p+1}$	
	
	$\lim\limits_{n \to \infty} \frac{S_n(p)}{n^{p+1}} = \lim\limits_{n \to \infty} \sum\limits_{k=1}^{n}\frac1n(\frac{k}{n})^p = \int\limits_0^1 f(t) \,dt$	
	
	$f: [0,1] \to \R \;\; f(t) = t^p$
	
	$\xi_k = \frac{k}{n}$
	
	Мелкость дробления $\frac1n \to 0$.
	
	$\implies\frac{S_n(p)}{n^{p+1}} \to \int\limits_0^1 t^p \,dt = \frac{1}{p+1}\implies S_n(p) \underset{n \to \infty}{\sim} \frac{n^{p + 1}}{p + 1}$	

	
	При $p=-1$ считаем, что $\frac1{p+1} = \infty$.
\end{example}

\begin{lemma}\slashns

    $f \in C^2[a,b]$. Тогда:

	$\int\limits_{\alpha}^{\beta} f(t) \, dt - \frac{f(\alpha) + f(\beta)}{2}(\beta - \alpha) = -\frac12 \int\limits_{\alpha}^{\beta} f''(t)(t-\alpha)(\beta - t) \, dt$	
\end{lemma}

\begin{proof}\slashns

	$\\ \gamma:= \frac{\alpha + \beta}{2}$
	
	$\int\limits_{\alpha}^{\beta} f(t) \, dt = \int\limits_{\alpha}^{\beta} f(t) (t - \gamma)'\, dt =  f(t)(t-\gamma) \Big|_{\alpha}^{\beta} - \int\limits_{\alpha}^{\beta} f'(t) (t - \gamma)\, dt
	= 
	f(\beta)(\beta-\gamma) - f(\alpha)(\alpha-\gamma) - \int\limits_{\alpha}^{\beta} f'(t) (t - \gamma)\, dt
	=\frac{f(\beta) + f(\alpha)}{2}(\beta - \alpha)- \int\limits_{\alpha}^{\beta} f'(t) (t - \frac{\alpha+\beta}{2})\, dt$
	
	$((t-\alpha)(\beta - t))' = \alpha + \beta - 2t = -2(t - \gamma)$
	
	$\Delta = -\int\limits_{\alpha}^{\beta} f'(t)(t - \gamma)\, dt
	=
	-\int\limits_{\alpha}^{\beta} f'(t)(-\frac12)((t-\alpha)(\beta - t))'\, dt
	=\frac12\int\limits_{\alpha}^{\beta} f'(t)((t-\alpha)(\beta - t))'\, dt
	=\\=\frac12 f'(t)(t-\alpha)(\beta - t) \Big|_{\alpha}^{\beta} - \frac12 \int\limits_{\alpha}^{\beta} f''(t)(t-\alpha)(\beta - t) \, dt
	= -\frac12 \int\limits_{\alpha}^{\beta} f''(t)(t-\alpha)(\beta - t) \, dt$
\end{proof}

\begin{theorem}[оценка погрешности в ф-ле трапеций]\slashns
	
	$f \in C^2[a,b]$ и $\tau$-- дробление. Тогда:
	
	$\left|\int\limits_a^b f - \sum\limits_{k=1}^{n} \frac{f(x_{k-1}) + f(x_k)}{2} (x_k - x_{k-1})\right| \le \frac{\left|\tau\right|^2}{8} \int\limits_a^b \left| f''\right|$
	
	В частности, если дробление на равные отрезки
	
	$\left|\int\limits_a^b f - \frac{b-a}{n}(\frac{f(x_0)}{2} + \sum\limits_{k=1}^{n-1}f(x_k) + \frac{f(x_n)}{2})\right| S\le \frac{(b-a)^2}{8n^2} \int\limits_a^b \left|f''\right|$
\end{theorem}


\begin{proof}\slashns
	
	$\Delta:= \int\limits_a^b f - \sum\limits_{k=1}^{n} \frac{f(x_{k-1}) + f(x_k)}{2}(x_k - x_{k-1}) = \sum\limits_{k=1}^{n}(\int\limits_{x_{k-1}}^{x_k} f - \frac{f(x_{k-1}) + f(x_k)}{2} (x_k - x_{k-1})) =\\= -\frac12 \sum\limits_{k=1}^n \int\limits_{x_{k-1}}^{x_k} f''(t)(t-x_{k-1})(x_k - t) \, dt $
	
	$\left|t-x_{k-1}\right|\left|x_{k}-t\right|\le \frac{(x_k - x_{k - 1})^2}{4}\le \frac{\left|\tau\right|^2}{4}$
	
	Первое неравенство выполняется, поскольку $a = \left|t-x_{k-1}\right|, b = \left|x_{k}-t\right|.\ t\in [x_k, x_{k-1}].\ a + b =  \left|x_k - x_{k - 1}\right|\\ \implies ab$ максимально при $a = b = \frac{(x_k - x_{k - 1})}{2}$(чтобы доказать это можно найти вершину параболы $\\y = (t-x_{k-1})(t-x_{k})$).
	
	$\left|\Delta\right| \le \frac12 \sum\limits_{k=1}^n \int\limits_{x_{k-1}}^{x_k} \left|f''(t)\right|(t - x_{k-1})(x_k - t) \, dt \le \frac12 \sum\limits_{k=1}^n \int\limits_{x_{k-1}}^{x_k} \left|f''(t)\right|\frac{\left|\tau\right|^2}{4} \, dt = \frac{\left|\tau\right|^2}{8} \int\limits_a^b \left|f''\right|$
		
	При делении на равные отрезки $\left|\tau\right| = \frac{b - a}{n}.$	
	
\end{proof}

