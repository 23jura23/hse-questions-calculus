\epigraph{А разве можно всё упростить, всё обобщить? И вообще, разве по чужому желанию можно обобщать и упрощать?}{Джером Дэвид Сэлинджер, "Над пропастью во ржи"}

Привет, Путник! Я рад сопровождать тебя в начале твоего долгого и тяжёлого пути к (не) отчислению. Запасись терпениеим. А лучше корвалолом.

\Subsection{Билет 1: ! Дробление, ранг, оснащение, сумма Римана.}

\begin{definition} \thmslashn 

	Дробление отрезка $[a,b]$ -- это набор точек $\tau$, такой что
	
	$\tau = \{x_k\}^n_{k=0} : a = x_0 < x_1 < x_2 < ... < x_n = b$

	Ранг (мелкость) дробления -- $\max\limits_{k=0}^{n-1}(x_{k+1} - x_{k}) = |\tau|$ 

	Оснащение -- набор точек, такой что

	$\{\xi_k\}_{k=0}^{n-1} : \xi_i \in [x_i, x_{i+1}]$

	Пара $(\tau, \xi)$ -- оснащённое дробление

\end{definition}

\begin{definition} \thmslashn 

	Сумма Римана (интегральная сумма)

	$f : [a,b] \mapsto R$ и оснащённое дробление $(\tau, \xi)$

	$S(f,\tau,\xi) = \sum\limits_{k=0}^{n-1} f(\xi_k)(x_{k+1} - x_{k})$

\end{definition}


Какой короткий и классный билет :)

Ну, удачи...
