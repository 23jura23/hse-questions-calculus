\Subsection{Билет 10: Абсолютная сходимость. Признак Дирихле.}

\begin{definition}[Абсолютная сходимость.] \thmslashn
	
	$f \in C[a,b)$
	
	$\int\limits_a^b f$ абсолютно сходится, если $\int\limits_a^b |f|$ сходится.
	
\end{definition}

\begin{theorem} \thmslashn
	
	Если $\int\limits_a^b f$ абсолютно сходится, то $\int\limits_a^b f$ сходится.
	
\end{theorem} 

\begin{proof} \thmslashn
    
    $ 0 \le f_{\pm} \le |f|$
    
    $\int\limits_a^b f$ абсолютно сходится $\implies \int\limits_a^b |f|$ сходится $\implies \int\limits_a^b f_{\pm}$ сходится
    
    $\int\limits_a^b f = \int\limits_a^b (f_{+} - f_{-}) = \int\limits_a^b f_{+} - \int\limits_a^b f_{-} \implies \int\limits_a^b f$ сходится.
    
\end{proof}

\begin{theorem}[признак Дирихле] \thmslashn
	
	$f, g \in C[a,+\infty)$
	
	1. $\exists M : \;\; |\int\limits_a^c f| \le M$ при всех $c > a$.
	
	2. $g$ -- монотонная функция.
	
	3. $\lim\limits_{x \to +\infty} g(x) = 0$
	
	Тогда $\int\limits_a^{+\infty} fg$ сходится.
\end{theorem}

\begin{proof} \thmslashn

    Лишь для $g \in C^1[a,+\infty)$.
    
    Пусть $F(y) := \int\limits_a^y f$
    
    По условию $|F| \le M$
    
    $\int\limits_a^c fg = \int\limits_a^c F'g =  \left. Fg \right|_a^c - \int\limits_a^c Fg'$
    
    Надо доказать, что существует предел при  $c \to +\infty$
    
    Распишем первое слагаемое как: $F(c)g(c) - F(a)g(a)$. Тогда $F(c)g(c) \to 0$ при $c \to +\infty$, так как это произведение бесконечно малой на ограниченную.
    
    Надо доказать, что $\int\limits_a^c Fg'$ сходится. Докажем, что он абсолютно сходится, то есть, что $\int\limits_a^c |F| \cdot |g'|$ сходится. 
    
    $\int\limits_a^c |F| \cdot |g'| \le M \int\limits_a^c |g'| = M |\int\limits_a^c g'| = M | \left. g \right|_a^c | = M|g(c) - g(a)| \le M|g(a)| \implies \int\limits_a^{+\infty} |F'g|$ сходится.
    
    
\end{proof}

