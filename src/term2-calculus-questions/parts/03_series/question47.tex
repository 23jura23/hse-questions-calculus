\Subsection{Билет 47: ! Признак Лейбница. Оценка суммы знакочередующегося ряда. Примеры (ряд Лейбница и
его перестановка)}

Знакочередующийся ряд: $\sum_{k=1}^{\infty} (-1)^{n-1}a_n$, $a_n \ge 0$

\begin{theorem}[Признак Лейбница]\thmslashn

Если $a_n$ монотонно убывают стремятся к 0, то ряд $\sum_{k=1}^{\infty} (-1)^{n-1}a_n$ сходится. Важно заметить, что условие стремления $a_n$ к 0 важно (см. необходимое условие сходимости ряда). Данный признак можно так же вывести из признака Дирихле. Однако мы хотм произвести так же оценку на сумму знакочередующегося ряда: $S_{2n} \le S \le S_{2n+1}$
\end{theorem}

\begin{proof}\thmslashn

$S_{2n+2}=S_{2n}+a_{2n+1}-a_{2n+2} \ge S_{2n}$
 $S_{2n+1}=S_{2n-1}-a_{2n}+a_{2n+1} \le S_{2n-1}$. Получаем, что $S$ c четными номерами растут, а с нечетными - убывают. Значит, мы можем расписать вложенную последовательно	сть отрезков: $[0,S_1] \supset [S_2,S_3] \supset [S_4,S_5]... \supset [S_{2n}, S_{2n+1}]$.\\ Теперь рассмотрим длины отрезков: $S_{2n+1}-S_{2n}=a_{2n+1} \to 0$. Тогда последовательность отрезков стягивается. Тогда применим одноименную теорему: у этих отрезков есть общая точка $S$, которая является пределом их концов: $\lim S_{2n} = \lim S_{2n+1} = S$. Получили что частичные суммы сходятся к $S$, значит исходный ряд тоже сходится к $S$. Также можно заметить, что неравенство на сумму ряда выполняется, потому что точка  $S$ лежит во всех отрезках, в частности в отрезке $[S_{2n}, S_{2n+1}]$\\
\end{proof}

\begin{example}\thmslashn

 В качестве примера рассмотрим ряд Лейбница: $\sum_{n=1}^\infty \frac{(-1)^{n-1}}{n} $ \\
 $S_{2n} = (1+ \frac{1}{3} + \frac{1}{5} + ... + \frac{1}{2n-1})-(\frac{1}{2}+ \frac{1}{4}+... + \frac{1}{2n}) = 1+ \frac{1}{2}+ \frac{1}{3}+ \frac{1}{2n} -2(\frac{1}{2}+\frac{1}{4}+...+ \frac{1}{2n}=H_{2n}-H_n=$ (сложили все дроби и дважды вычли четные), где $H_n$- гармонический ряд (смотри билет 6). Подставим все в формулу для гармонических чисел: $=\ln(2n)+ \gamma +o(1)-(\ln(n) + \gamma + o(1)) = \ln(2n) - \ln(n) + o(1) = \ln 2 + o(1)$. Тогда $S_{2n} \to \ln 2$. Тогда $\sum_{n=1}^\infty \frac{(-1)^{n-1}}{n}= \ln 2 $. \\
 Рассмотрим перестановку ряда Лейбница: $(1-\frac{1}{2}-\frac{1}{4})+(\frac{1}{3}-\frac{1}{6}-\frac{1}{8})+...+(\frac{1}{2n-1}-\frac{1}{4n-2}-\frac{1}{4n})+...$.
Будем считать $S_{3n} = 1+\frac{1}{3} + ... +\frac{1}{2n-1} -(\frac{1}{2}+\frac{1}{4}+...+\frac{1}{4n} = 1+\frac{1}{3} + ... +\frac{1}{2n-1} -\frac{1}{2}(1+\frac{1}{2}+\frac{1}{3}+...+\frac{1}{2n}=$. Последнее выражение в скобках это $H_{2n}$. Мы только что считали это в предыдущем примере. Получаем: $=H_{2n}-\frac{1}{2}H_n=\frac{\ln 2}{2}$
\end{example}
