\Subsection{Билет 48: Перестановка членов абсолютно сходящегося ряда}


\begin{definition} \thmslashn 

Перестановка членов ряда: $\varphi: \N \to \N$ - биекция и $\sum a_n$ - исходный ряд. Тогда $\sum a_{\varphi(n)}$ - перестановка члена ряда.\\
\end{definition}

\begin{theorem}\thmslashn

Если $\sum a_n$ абсолютно сходится к $S$, то перестановка ряда $\sum a_{\varphi(n)} $ сходится,причем, тоже к $S$.
\end{theorem}

\begin{proof}\thmslashn

 Случай 1 $a_n \ge 0$. Также введем обозначение $S' =\sum_{k=1}^n a_{\varphi(k)} $, а $S = \sum_ {k=1}^n a_k$. Тогда мы точно знаем, что $S'_n \ge S$, так как в сумме $S'$ встречаются не все слагаемые, а те, которые отсутствуют $\ge 0$, поэтому сумму они только увеличивают. Тогда $\lim S'_n = S' \leq S$, то есть $S' \le S$. Так как у нас биекция - мы можем сделать обратную перестановку, от которой сумма ряда не увеличится. Сделаем перестановку туда и обратно и получим, что каждая из них не увеличивает сумму ряда, ну значит эти суммы равны между собой: $S'=S$. \\
Случай 2: $a_n \in \mathbb{R}$: заведем $a_n(+) = max \{a_n,0\}$ и $a_n(-)=max \{-a_n,0\}$. $a_n(+)-a_n(-)=a_n$, $a_n(+)+a_n(-)=|a_n|$
Так как по условию $\sum |a_n|$ сходится абсолютно, то $\sum a_n(\pm)$ сходится. Более того ряды - с неотрицательными слагаемыми, значит, перестановка членов не меняет суммы ряда, значит $\sum a_{\varphi(n)}(\pm) = \sum a_n(\pm)$. Тогда $\sum a_{\varphi(n)} = \sum a_{\varphi(n)}(+)-\sum a_{\varphi(n)}(-)=\sum a_n (+) - \sum a_n(-)= \sum a_n$\\
\end{proof}

\begin{remark}\thmslashn

	\begin{enumerate}

		\item Если $a_n \ge 0$ и ряд расходится, то перестановка ряда так же расходится. Это верно, так как если бы нашлась перестановка, дающая сходящийся ряд, тогда бы обратная перестановка тоже давала бы сходящийся ряд, а это противоречит тому, что исходный ряд расходится.\\
		\item Другое замечание : если $\sum a_n$ сходится условно, то $\sum a_n(\pm)$ расходятся. Так как $\sum a_n = \sum a_n (+) - \sum a_n(-)$. Если бы один из них сходился, то сходился бы и другой, так как один выражается через другой с помощью $\sum a_n$, который сходящийся. Ну тогда ряд $\sum |a_n| = \sum a_n (+) + \sum a_n(-)$ тоже бы сходился, как сумма сходящихся. Пришли к противоречию.
	\end{enumerate}
\end{remark}