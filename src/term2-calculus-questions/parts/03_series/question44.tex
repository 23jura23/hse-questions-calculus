\Subsection{Билет 44: Признак Даламбера. Примеры. Связь между признаками Коши и Даламбера.}

\begin{theorem}[признак Даламбера]\slashns
	
	$a_n > 0$
	
	\begin{enumerate}
		\item Если $\frac{a_{n+1}}{a_n} \ge 1$, то ряд расходится.
		\item Если $\frac{a_{n+1}}{a_n} \le d < 1$, то ряд сходится.
		\item Пусть $d^* = \lim\limits_{n \to \infty} \frac{a_{n+1}}{a_n}$
		
		Если $d^* < 1$, то ряд сходится.
		
		Если $d^* > 1$, то ряд расходится.
		
	\end{enumerate}
\end{theorem}

\begin{proof}\slashns
	
	\begin{enumerate}
		\item $\frac{a_{n+1}}{a_n} \ge 1 \implies a_n \le a_{n+1}$

		$ \implies$ начиная с некоторого места члены ряда возрастают, $0 < a_1 \le a_2 \le ...$
		
		$\implies a_n \not\to 0 \implies$  ряд расходится

		\item $\frac{a_{n+1}}{a_n} \le d \implies a_{n+1} \le d \cdot a_{n}$ начиная с некоторого места.

		$\implies a_{n+k} \le d^k \cdot a_n$ при всех $k$

		$\implies$ при всех $k \ge n$ $a_k \le d^{k - n} \cdot a_n = d^k \cdot \frac{a_n}{d^n} = d^k \cdot const$

		$\implies a_k = \mathcal{O}(d^k)$

		$\sum\limits_{n=1}^{\infty}d^k$ -- это сумма геометрической прогрессии, знаменатель которой меньше 1, то есть $\sum\limits_{n=1}^{\infty}d^k$ сходится, тогда по признаку сравнения $\sum\limits_{n=1}^{\infty}a_n$ тоже сходится.
		
		\item 
		\begin{enumerate}
			\item $\lim\limits_{n \to \infty} \frac{a_{n+1}}{a_n} = d^* < 1$
		
			$d:= \frac{d^*+1}{2} < 1$
		
			$\implies$ начиная с некоторого номера $\frac{a_{n+1}}{a_n} < d < 1 \implies$ попали в первый пункт, сходится
		
			\item $\lim\limits_{n \to \infty} \frac{a_{n+1}}{a_n} = d^* > 1$
		
			$\implies$ с некоторого номера $\frac{a_{n+1}}{a_n} \ge 1$
		
			$\implies$ ряд расходится.
		\end{enumerate} 
	\end{enumerate}
\end{proof}

\begin{remark}\slashns
	
	Если $\lim\limits_{n \to \infty} \frac{a_{n+1}}{a_n} = 1$, то ряд может как сходиться, так и расходиться.

	\begin{enumerate}

		\item $\sum\limits_{n = 1}^{\infty} \frac1n$ -- расходится

		$\lim\limits_{n \to \infty} \frac{\frac 1{n+1}}{\frac 1n} = \lim\limits_{n \to \infty} \frac n{n+1} = 1$

		\item $\sum\limits_{n = 1}^{\infty} \frac1{n(n+1)}$ -- сходится
	
		$\lim\limits_{n \to \infty} \frac{\frac 1{(n+1)(n+2)}}{\frac 1{n(n+1)}} = \lim\limits_{n \to \infty} \frac n{n+2} = 1$
	\end{enumerate}
\end{remark}

\begin{example}\slashns
	
	$\sum\limits_{n = 0}^{\infty} \frac{x^n}{n!}$, при $x > 0$
	
	\underline{По Даламберу.}

	 $\frac{a_{n+1}}{a_n} = \frac{x^{n+1}}{(n+1)!} \cdot \frac{n!}{x^n} = \frac{x}{n+1} \to 0$
	
	$\implies $ ряд сходится
	
	\underline {По Коши.}
	
	$\sqrt[n]{\frac{x^n}{n!}} = \frac{x}{\sqrt[n]{n!}}$ 

	Воспользуемся формулой Стирлинга:

	$ \frac{x}{\sqrt[n]{n!}} \sim \frac{x}{\sqrt[n]{n^n\cdot e^{-n} \cdot \sqrt{2\pi n}}}\sim \frac x {\frac ne} = \frac{xe}{n} \to 0$

	$\implies $ ряд сходится
\end{example}

\begin{theorem}[Связь между признаками Коши и Даламбера]\slashns
	
	$a_n > 0$
	
	Если существует $\lim\limits_{n \to \infty} \frac{a_{n+1}}{a_n} =: d$, то существует и $\lim\limits_{n \to \infty} \sqrt[n]{a_n}$ и он равен $d$
\end{theorem}

\begin{proof}\slashns
	
	Будем рассматривать не сами выражения, а их логарифмы.

	$\lim\limits_{n \to \infty} \frac{a_{n+1}}{a_n} = d \iff \lim\limits_{n \to \infty} \ln(\frac{a_{n+1}}{a_n}) = \ln d$

	Хотим доказать, что $\lim\limits_{n \to \infty} \ln \sqrt[n]{a_n} = \ln d$
	

	Применяем Штольца!
	
	$\lim\limits_{n \to \infty} \ln \sqrt[n]{a_n} = \lim\limits_{n \to \infty} \frac{\ln a_n}{n} = \lim\limits_{n \to \infty} \frac{\ln a_{n+1} - \ln a_n}{(n+1) - n} = \lim\limits_{n \to \infty} \ln (\frac{a_{n+1}}{a_n}) = \ln d$
	
\end{proof}

