\Subsection{Билет 49: Теорема Римана.}
\begin{theorem}[Римана]  \thmslashn
	
	$a_n \in \R$  $\sum a_n$ условно сходится.
	
	Тогда для любого $S \in \overline{\R}$ существует перестановка $\phi$, т.ч. $\sum\limits_{n = 1}^{\infty}a_{\phi(n)} = S$. Также существует перестановка $\phi$, для которой ряд не имеет суммы.
\end{theorem}

\begin{proof}  \thmslashn
	
	$\sum b_n$ и $\sum c_n$ -- ряды $\sum (a_n)_{\pm}$, из которых выкинули все нули.
	
	$\sum b_n$ и $\sum c_n$ -- расходятся (т.к. есть условная сходимость), Более того, $\sum b_n = \sum c_n = +\infty$. При этом $\lim b_n = \lim c_n = 0$ (необходимое условие сходимости для ряда $\sum a_n$).
	
	Пункты $a), b), c)$ доказываются аналогично. Наверное, можно на экзамене расписать только пункт $a)$, а про остальные сказать, что аналогично. Здесь на всякий случай расписаны все три пункта.
	\begin{enumerate}
	
	\item[a)] Пусть $S \in \R$. Будем набирать частичную сумму так, чтобы она поочередно превышала $S$ и наоборот была меньше $S$. Мы можем это сделать, т.к. $\sum b_n = \sum c_n = +\infty$.
	
	$b_1+b_2+...+b_{n_1 - 1} \le S < b_1 + b_2 + ... + b_{n_1}$ 
	
	$b_1+b_2+...+b_{n_1} -c_1-...-c_{m_1} < S\le b_1+b_2+...+b_{n_1}-c_1-...-c_{m_1 - 1}$
	
	$b_1+...+b_{n_1}-c_1-...-c_{m_1} + b_{n_1+1} + ...+b_{n_2 - 1} \le S < b_1+...+b_{n_1}-c_1-...-c_{m_1} + b_{n_1+1} + ...+b_{n_2}$
	
	$b_1+...+b_{n_1}-c_1-...-c_{m_1} + b_{n_1+1} + ...+b_{n_2} - c_{m_1+1}-...-c_{m_2} < S \le\\\le b_1+...+b_{n_1}-c_1-...-c_{m_1} + b_{n_1+1} + ...+b_{n_2}-c_{m_1+1}-...-c_{m_2-1}$
	
	
	И так далее.
	
	$|\text{частичная сумма} - S| \le |\text{последнего взятого элемента}| \to 0$. Значит частичная сумма построенного ряда $\to S$.
	
	\item[b)] Пусть $S = +\infty$. Мы знаем, что $\sum b_n = +\infty$. Поэтому мы можем нашу перестановку получить следующим образом:
	
	$b_1 + b_2 + ...+b_{n_1} > 1 \ge b_1+b_2+...+b_{n_1 - 1}$ (раз $\sum b_n = +\infty$, то в какой-то момент сумма превысит 1)
	
	$b_1 +...+b_{n_1} + c_1$ (добавили элемент из ряда $c_n$)
	
	$b_1 +...+b_{n_1} + c_1 + b_{n_1+1} + ...+b_{n_2 } > 2 \ge b_1 +...+b_{n_1} + c_1 + b_{n_1+1} + ...+b_{n_2 -1}$
	
	И так далее.
	
	\item[c)] Пусть мы хотим получить перестановку $\phi$, для которой ряд не имеет суммы. Будем набирать суммы так, чтобы она то была больше 1, то меньше -1. Это опять же можно сделать, т.к. $\sum b_n = \sum c_n = +\infty$.
	
	$b_1+b_2+...+b_{n_1 - 1} \le 1 < b_1 + b_2 + ... + b_{n_1}$ 
	
	$b_1+b_2+...+b_{n_1} -c_1-...-c_{m_1} < -1 \le b_1+b_2+...+b_{n_1}-c_1-...-c_{m_1 - 1}$
	
	$b_1+...+b_{n_1}-c_1-...-c_{m_1} + b_{n_1+1} + ...+b_{n_2 - 1} \le 1 < b_1+...+b_{n_1}-c_1-...-c_{m_1} + b_{n_1+1} + ...+b_{n_2}$
	
	$b_1+...+b_{n_1}-c_1-...-c_{m_1} + b_{n_1+1} + ...+b_{n_2} - c_{m_1+1}-...-c_{m_2} < -1 \le\\\le b_1+...+b_{n_1}-c_1-...-c_{m_1} + b_{n_1+1} + ...+b_{n_2}-c_{m_1+1}-...-c_{m_2-1}$
	
	И так далее.
	\end{enumerate}
\end{proof}