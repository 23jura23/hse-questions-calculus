\Subsection{Билет 70: Формула для коэффициентов разложения в ряд аналитической функции. Несовпадение классов бесконечно дифференцируемых и аналитических функций.}

\begin{theorem}[единственность разложения функции в степенной ряд] \thmslashn
	
	Пусть $f(z) = \sum\limits_{n=0}^{\infty} a_n(z - z_0)^n$ при $|z-z_0| < R$ -- радиус сходимости.
	
	Тогда ряд раскладывается единственным образом, причем коэффициенты в этом ряду будут выглядеть так: $a_n = \frac{f^{(n)}(z_0)}{n!}$
	\begin{proof} \thmslashn
	
		По предыдущей теореме:
		\[
		f^{(m)}(z) = \sum\limits_{n=m}^{\infty} n(n-1)\ldots(n-m+1)a_n(z-z_0)^{n-m}
		\]
		Подставим $z = z_0$. Тогда все слагаемые кроме первого занулятся и получим: 
		\[
		f^{(m)}(z_0) = m(m-1)\ldots1\cdot a_m = m!a_m
		\].
		Отсюда $a_m = \frac{f^{(n)}(z_0)}{n!}$.
	\end{proof}
\end{theorem}

\begin{definition} \thmslashn
	
	\textcolor{red}{\textbf{Ряд Тейлора}} функции $f$ в точке $z_0$ называется ряд $\sum\limits_{n=0}^{\infty} \frac{f^{(n)}(z_0)}{n!}(z-z_0)^n$
\end{definition}

\begin{definition} \thmslashn
	
	Функция называется \textcolor{red}{\textbf{аналитической}} в точке $z_0$, если она является суммой своего ряда Тейлора для точки $z_0$ в окрестности точки $z_0$.
\end{definition}

Ряд Тейлора мы можем писать только, если функция бесконечно дифферинцируема. Но бывают бесконечно дифференцируемые функции, которые не являются аналитическими, например:

\begin{example} \thmslashn
	
	\[
	f(x) = \begin{cases}
   		e^{-1/x^2} &\text{при $x \neq 0$}\\
   		0 &\text{при $x = 0$}
	 \end{cases}
	\]
	Рассмотрим точки $x \neq 0$:
	\[
	f^{(n)}(x) = \frac{P_n(x)}{x^{3n}}e^{-1/x^2}
	\]
	Идем по индукции $(n \to n + 1)$, проверяем есть ли формула для разных производных:
	\begin{description}
		\item[База:]
			Для $f$: $f = P_0e^{-1/x^2}$, то есть $P_0 \equiv 1$
		\item[Переход:]
			\[
			f^{(n+1)}(x) = \left( f^{(n)}(x)\right)' = (P_n(x)x^{-3n}e^{-1/x^2})' =
			\]
			\[
			= P_n(x)x^{-3n}e^{-1/x^2}\frac{1}{x^3} + P_n'(x)x^{-3n}e^{-1/x^2}+P_n(x)(-3n)x^{-3n-1}e^{-1/x^2} = \frac{e^{-1/x^2}}{x^{3n+3}}P_{n+1}(x)
			\]
	\end{description}
	
	Найдем $f^{(n)}(0) = \lim\limits_{x \to 0} \frac{f^{(n-1)}(x) - f^{(n-1)}(0)}{x}$
	Докажем по индукции $(n-1 \to n)$, что $f^{(n)}(0) = 0$.
	\begin{description}
		\item[Переход:]
			\[
			f^{(n)}(0) = \lim\limits_{x \to 0} \frac{f^{(n-1)}(x) - f^{(n-1)}(0)}{x} = \lim\limits_{x \to 0}\frac{f^{(n-1)}}{x} =  \lim\limits_{x \to 0}e^{-1/x^2}\frac{P_n(x)}{x^{3n+1}} \underset{y = 1/x}{=} \lim\limits_{y \to \infty} e^{-y^2}y^{3n+1}P_n\left( \frac{1}{y}\right) = 0
			\]
			
			$P_n\left( \frac{1}{y}\right) \xrightarrow[y \to \infty]{} P_n(0)$ -- константа
			
			$e^{-y^2}y^{3n+1} \xrightarrow[y \to \infty]{} 0$, так как $e^{-y^2}$ убывает быстрее.
	\end{description}
	
	Значит ряд Тейлора равен 0, но функция не 0 в точках $x \neq 0$. Значит функция не аналитическая.
\end{example}