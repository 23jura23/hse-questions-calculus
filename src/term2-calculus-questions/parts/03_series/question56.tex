\Subsection{Билет 56: Пространство $\ell^{\infty}$ и его полнота}

\begin{definition} Пространство $\ell^{\infty}(E)$. \thmslashn
	
	\begin{center}
		$\ell^{\infty}(E) := \{ f : E \mapsto \R $ $|$ $\sup\limits_{x \in E}$$|f(x)| < +\infty \}$
	\end{center}
	
	c нормой $||f||_{\ell^{\infty}(E)} := \sup\limits_{x \in E}|f(x)|$
\end{definition}

Другими словами, нормированное пространство $\ell^{\infty}(E)$ состоит из ограниченных на $E$ функций.

\begin{remark} $\sup\limits_{x \in E}|f(x)|$ действительно норма \thmslashn
	
	\begin{enumerate}
	
		\item $||f|| \ge 0$ и $||f|| = 0$ $\iff$ $\sup\limits_{x \in E}|f(x)| \ge 0$ и $\sup\limits_{x \in E}|f(x)| = 0$ $\iff$ $f \equiv 0$
	
		\item $\sup\limits_{x \in E}|\lambda f(x)| = {|\lambda| \sup\limits_{x \in E}f(x)} \quad \forall {\lambda \in \R} \quad \forall x \in X$
	
		\item Неравенство треугольника
	
			$||f + g|| = \sup\limits_{x \in E}|f(x) + g(x)| \le \sup\limits_{x \in E}(|f(x)| + |g(x)|) \le \sup\limits_{x \in E}|f(x)| + \sup\limits_{x \in E}|g(x)| = ||f|| + ||g||$
	
	\end{enumerate}
	
	В доказательстве нер-ва треугольника пользовались тем, что $|a+b| \le |a| + |b|$ и $\sup(f+g) \le \sup(f) + \sup(g)$
	
\end{remark}

\begin{remark} Связь нормы с равномерной сходимостью \thmslashn
	
	$f_n \uniconv f \text{ на } E \iff \lim\limits_{n \to \infty}\sup\limits_{x \in E}|f_n(x) - f(x)| = 0 \iff \lim\limits_{n \to +\infty}||f_n - f|| = 0 \iff$
	 
	$f_n \text{ сходится к } f \text{ в пространстве } \ell^{\infty}(E) $
	
	То есть про равномерную сходимость можно думать как про сходимость в специальном нормированном пространстве.
	
\end{remark}

\begin{theorem} \thmslashn
	
	$\ell^{\infty}(E)$ - полное нормированное пространство.
	
	\begin{proof} \thmslashn
		
		Надо доказать, что каждая фундаментальная последовательность из $\ell^{\infty}$ сходится к элементу этого же пространства.
		
		Пусть $f_n$ фундаментальная последовательность из $\ell^{\infty}$. Тогда
		\[\forall {\eps > 0} \quad {\exists N} \quad \forall {m,n \ge N} \quad || f_n - f_m|| < \eps \]
		
		Заметим, что $|| f_n - f_m || = \sup\limits_{x \in E}|f_n(x) - f_m(x) | \ge |f_n(x) - f_m(x)|$ при $x \in E$.
		
		То есть $|f_n(x) - f_m(x)| < \eps$. Тогда по критерию Коши для равномерной сходимости $f_n \uniconv f$, где $f : E \mapsto \R$ - некоторая функция.
		
		Осталось понять, что $f \in \ell^{\infty}(E)$, т.е. что $f$ - ограниченная функция.
		
		Подставим $\eps = 1$ в определение равномерной сходимости. Для него найдется $N$, т.ч. при $n \ge N$ $|f_n(x) - f(x)| < 1$ при всех $x \in E$. Тогда по неравенству треугольника : 
		\[|f(x)| \le |f_n(x)| + |f(x) - f_n(x)| < |f_n(x)| + 1 \le ||f_n|| + 1\]
		
		Но т.к. $n$ - фиксированное число, то $|f(x)|$ не превосходит какого-то фиксированного выражения. Значит $f$ - ограниченная функция.
 	\end{proof}
\end{theorem}