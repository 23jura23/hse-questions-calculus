\Subsection{Билет 66: Степенные ряды. Теорема о сходимости
    ряда при меньших аргументах. Радиус и круг схо-
    димости. Формула Коши–Адамара. Примеры.}


\begin{definition}\slashns
    
    Степенной ряд с центром в $z_0$:
    
    $\sum\limits_{n = 0}^{\infty} a_n(z - z_0)^n \;\; a_n,z_0, z \in \CC$
    
    Мы всегда можем выбрать точку $w := z - z_0$, тогда у нас всегда центр будет в точке 0.
    
    $\sum\limits_{n = 0}^{\infty} a_nw^n$
\end{definition}

\begin{theorem}\slashns
    
    Если $\sum\limits_{n = 0}^{\infty} a_nz^n$ сходится при $z = z_0 \not = 0$, то он абсолютно сходится и при всех $\abs{z} < \abs{z_0}$.
\end{theorem}

\begin{proof}\slashns
    
    $\sum a_nz_0^n$ сходится $\Rightarrow a_nz_0^n \to 0$, значит $\abs{a_nz_0^n} \leqslant M \forall n$.
    
    $\abs{a_n z^n} = \abs{ a_nz_0^n (\frac{z}{z_0})^n } = \abs{ a_nz_0^n} \abs{ \frac{z}{z_0} }^n \leqslant M \cdot \abs{ \frac{z}{z_0} }^n$ -- этот ряд абсолютно сходится, т.к. это геометрическая прогрессия.

\end{proof}

\begin{consequence}\slashns
    
    $\sum a_nz^n$ расходится при $z = z_0$, то он расходится и при $\abs{z} > \abs{z_0}$
\end{consequence}

\begin{proof}\slashns
    
    От противного. Допустим он сходится в $\abs{z} > \abs{z_0}$, тогда он сходится и в $z_0$.
\end{proof}

\begin{definition}\slashns
    
    Радиус сходимости степенного ряда -- такое число $R \in [0, +\infty]$, что при $\abs{z} < R$ ряд сходится, а при $\abs{z} > R$ ряд рассходится. (для рядов с центорм в точке 0, иначе $\abs{z - z_0} < R$ сходится и $\abs{z - z_0} > R$ расходится).
\end{definition}

\begin{definition}\slashns
    
    Круг сходимости -- круг радиуса $R$ с центром в точке $z_0$, где $R$ -- радиус сходимости.
\end{definition}

\begin{theorem}[Формула Коши-Адамара]\slashns
    
    Всякий степенной ряд имеет радиус сходимости и он выражается форулой $R = \frac{1}{\varlimsup\limits_{n \to \infty} \sqrt[n]{\abs{a_n}} }$
\end{theorem}

\begin{proof}\slashns
    
    Применим признак Коши к ряду.
    
    $K := \varlimsup\limits_{n \to \infty} \sqrt[n]{\abs{a_nz^n}} = \varlimsup\limits_{n \to \infty} \sqrt[n]{\abs{a_n}} \abs{z}$
    \begin{description} 
        \item 
        Если $K < 1$, то ряд абсолютно сходится $\varlimsup\limits_{n \to \infty} \sqrt[n]{\abs{a_n}} \abs{z} < 1 \Leftrightarrow \abs{z} < R$.
    
        \item 
        Если $K > 1$, то члены ряда не стремятся к 0 $\Rightarrow$ ряд расходится $\varlimsup\limits_{n \to \infty} \sqrt[n]{\abs{a_n}} \abs{z} > 1 \Leftrightarrow \abs{z} > R$.
    \end{description}
\end{proof}

\begin{remark}\slashns
    
    Внутри круга сходимости ряд сходится абсолютно. 
\end{remark}

\begin{example}\slashns
    
    \begin{enumerate}
        \item $\sum\limits_{n = 0}^{\infty} n!z^n \;\; R = 0$
        
        $\varlimsup \sqrt[n]{n!} = \lim \sqrt[n]{n^n e^{-n} \sqrt{2\pi n}} = \lim \frac{n}{e} \to +\infty$
        
        \item $\sum\limits_{n = 0}^{\infty} \frac{z^n}{n!} \;\; R = +\infty$
        
        $\varlimsup \sqrt[n]{\frac{1}{n!}} = \frac{1}{\lim \sqrt[n]{n!}} = 0$
        
    \end{enumerate}
\end{example}

