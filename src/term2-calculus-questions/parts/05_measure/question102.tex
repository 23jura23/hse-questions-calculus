\Subsection{Билет 102: ! Полукольца ячеек. Представление от- крытого множества в виде объединения ячеек. Следствие}

\begin{definition} \thmslashn

  $\mathcal{P}^m$ - все ячейки в $\mathbb{R}^m$
  
  $\mathcal{P}^m_{\mathbb{Q}}$ - все такие ячейки в $\mathbb{R}^m$, что их вершина в рациональных точках\\
\end{definition}
\begin{theorem} \thmslashn

  $\mathcal{P}^m$ и $\mathcal{P}^m_{\mathbb{Q}}$ - полукольца.
  \begin{proof} \thmslashn

    Понятно, что
    \begin{align*}
      \mathcal{P}^m &= \underbrace{\mathcal{P} \times \mathcal{P} \times ... \times \mathcal{P}}_m\\
      \mathcal{P}^m_{\mathbb{Q}} &= \underbrace{\mathcal{P}_{\mathbb{Q}} \times \mathcal{P}_{\mathbb{Q}} \times ... \times \mathcal{P}_{\mathbb{Q}}}_m
    \end{align*}
    
    $\mathcal{P}^m$ и $\mathcal{P}^m_{\mathbb{Q}}$ - полуинтервалы, про них уже знаем, что они - полукольца. Уже доказали, что декартово произведение полуколец - полукольцо. Несложно видеть, что из этого следует, что мы уже доказали теорему.
  \end{proof}
\end{theorem}
\begin{theorem} \thmslashn

  Всякое непустое открытое множество $G \subset R^n$ есть дизъюнктное объединение счетного числа ячеек таких, что их замыкания содержатся в $G$. Более того можно брать ячейки с рациональными вершинами.
  \begin{proof} \thmslashn

    Возьмем точку $x\in G$, она содержится там с каким-то шариком с центром в точке $x$ (ведь G открытое по условию). В этом шарике мы можем взять ячейку, которая содержит $x$, например, вписать туда кубик. Немного пошевелим его так, чтобы его вершины стали рациональными, он может и перестанет быть кубиком, но ячейкой он быть не престанет и все еще будет содержаться в шарике. Значит для каждой точки $x$ из $G$ есть такая ячейка $R_x$с рациональными, что $x \in R_x$, и $\text{Cl}R_x\subset G$. Но всего ячеек с рациональными вершинами счетное число (задается $2m$ рациональными точками по $m$ на вершину). Точек несчетное, а ячеек счетно, значит, будут повторяющиеся. Выкинем все повторы. Получили $G = \bigcup R_x$ (не по всем $x$, по счетному числу). Но по теореме (однйо из предыдущих) мы можем любое объединение превратить в дизъюнктное объединение, поэтому $G = \bigsqcup\bigsqcup Q$. (Рациональные концы никуда не делись, потому что мы там брали разность каких-то множеств с рациональными концами, так рациональными и осталось бы). Еще про замыкание: замыкание $Q$ содержится в объединение $R_x$, которое содержится в $G$.
  \end{proof}
\end{theorem}
\begin{consequence} \thmslashn

  $\mathcal{B}(\mathcal{P}_{\mathbb{Q}}^m) = \mathcal{B}(\mathcal{P}^m) = \mathcal{B}^m$
  \begin{proof} \thmslashn

    Покажем включения:

    $\mathcal{B}(\mathcal{P}_{\mathbb{Q}}^m) \subset \mathcal{B}(\mathcal{P}^m)$, ведь $\mathcal{P}_{\mathbb{Q}}^m\subset\mathcal{P}^m$, значит, в $\sigma$-алгебра, натянутая на правое, будет больше, чем $\sigma$-алгебра, натянутая на левое.

    $\mathcal{B}(\mathcal{P}^m) \subset \mathcal{B}^m$, ведь любая ячейка - счетное пересечение открытых множеств, а в $\mathcal{B}^m$ живут все объединения открытых множеств, значит, минимальная $\sigma$-алгебра, натянутая на ячейки лежит в $\mathcal{B}^m$.

    $\mathcal{B}^m \subset \mathcal{B}(\mathcal{P}_{\mathbb{Q}}^m)$, ведь любое открытое множество представляется как дизъюнктное объединение ячеек с рациональными концами, значит любое открытое множество лежит в $\mathcal{B}(\mathcal{P}_{\mathbb{Q}}^m)$, значит оно содержит минимальную $\sigma$-алгебру, содержащую все открытые множество - $\mathcal{B}$.
  \end{proof}
\end{consequence}
