\Subsection{Билет 15: Замкнутые множества: определение и свойства. Замыкание множества, связь со внутренностью.}
\begin{definition} \thmslashn 

    Пусть $\left<X, \rho\right>$ - метрическое пространство, $A \subset X$.

    $A$ называется замкнутым, если $X \setminus A$ - открыто. 
\end{definition}
\begin{properties} \thmslashn

    \begin{enumerate}
        \item $ \emptyset, X$ - замкнуты.
        \item Пересечение любого количества замкнутых множеств замкнуто
            \begin{proof} \thmslashn
            
                \begin{equation*}
                        X \setminus \bigcap_{\alpha\in I} A_{\alpha} = \bigcup_{\alpha\in I} (X \setminus A_{\alpha}) 
                \end{equation*}
            \end{proof}

            Так-как $\forall{\alpha}\quad X \setminus A_{\alpha}$ - открытое, то $\bigcup_{\alpha\in I} A_{\alpha}$ - откртоые, значит $\bigcap_{\alpha\in I} A_{\alpha}$ - замкнутое.

        \item Объединение конечного количества замкнутых множеств замкнуто
            \begin{proof} \thmslashn
            
                \begin{equation*}
                    X \setminus \bigcup_{k = 1}^{n} A_{k} = \bigcap_{k = 1}^{n} (X \setminus A_{k})
                \end{equation*}
            \end{proof}

            $X \setminus A_{k}$ открыто, значит их конечное пересечение открыто, значит $\bigcup_{k=1}^{n} A_{k}$ - замкнуто.
        \item $\forall{a\in X}\quad \forall{r > 0}\quad \overline{B}_{r}(a)$ - замкнутое множество.
            \begin{proof} \thmslashn
            
                Покажем что $X \setminus \overline{B}_{r}(a) = \{x\in X \ssep \rho(x, a) > r\} $ - открыто.

                Пусть $x\in X \setminus \overline{B}_{r}(a)$. $\tilde{r} = \rho(x, a) - r$. Тогда докажем что $B_{\tilde{r}}(x)\cap B_{r}(a) = \emptyset$:

                Пусть $y\in B_{\tilde{r}}(x)\cap \overline{B}_{r}(a)$, тогда $\rho(x, y) < \tilde{r}$, $\rho(y, a) < r$.
                \[ \rho(x, a) \overset{\triangle}{\le} \rho(x, y) + \rho(y, a) < \tilde{r} + r = \rho(x, a)  .\]

                Получили противоречие, значит $B_{\tilde{r}}(x)\cap B_{r}(a) = \emptyset \implies B_{\tilde{r}}(x) \subset X \setminus \overline{B}_{r}(a)$, значит $X \setminus \overline{B}_{r}(a)$ - открытое.
            \end{proof}
    \end{enumerate}
\end{properties}

\begin{definition} \thmslashn 

    Пусть $\left<X, \rho\right>$ - метрическое пространство.

    Замыкание множества $A \subset X$ - пересечение всех замкнутых множеств, содержащих $A$. Обозначается $\Cl A$ или $\overline{A}$.
\end{definition}

\begin{theorem} \thmslashn

    \[ \Cl A = X \setminus \Int (X \setminus A) .\]
    \begin{proof} \thmslashn
    
        Будем доказывать в виде $X \setminus \Cl A = \Int (X \setminus A)$ :

        Знаем, что $\Int (X \setminus A) = \bigcup_{\alpha} U_{\alpha}$ по всем $U_{\alpha}$ таким, что $U_{\alpha} \subset (X \setminus A)$ и $U_{\alpha}$ открыто.

        Пусть $C$ - замкнутое множество, такое, что $A \subset C$. Тогда $X \setminus C$ - открытое, и $(X \setminus A) \subset (X \setminus C) \implies \exists{\alpha}\quad U_{\alpha} = X \setminus C$.

        Аналогично в другую сторону - $\forall{\alpha}\quad X \setminus U_{\alpha}$ - замкнутое надмножество $A$.

        Пусть $C_{\alpha} = X \setminus U_{\alpha}$.

        \[ X \setminus \Cl A = X \setminus \bigcap_{\alpha} C_{\alpha} = \bigcup_{\alpha} (X \setminus C_{\alpha}) = \bigcup_{\alpha} U_{\alpha} = \Int (X \setminus A). \qedhere\] 


    \end{proof}
\end{theorem}
