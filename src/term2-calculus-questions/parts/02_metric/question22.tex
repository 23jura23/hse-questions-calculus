\Subsection{Билет 22: Покрытия. Компактность. Компактность в пространстве и в подпространстве. Простейшие свойства компактных множеств.}

\begin{definition} \thmslashn 

    Пусть $\left<X, \rho\right>$ - метрическое пространство.

    Семейство множеств $U_{\alpha} \subset X$ называется открытым покрытием множества $A$ (покрытием $A$ открытыми множествами), если
    \begin{enumerate}
        \item $A \subset \bigcup_{\alpha\in I} U_{\alpha} $ 
        \item $\forall{\alpha\in I}\quad U_{\alpha}$ - открытое.
    \end{enumerate}
\end{definition}
\begin{definition} \thmslashn 

    Пусть $\left<X, \rho\right>$ - метрическое пространство.

    $K \subset X$ называется компактом, если из любого отркытого покрытия можно выбрать конечное открытое покрытие.
\end{definition}
\begin{theorem} \thmslashn

    Пусть $\left<X, \rho\right>$ - метрическое пространтсво, $Y \subset X$ - подпространство.

    Тогда компактность $K \subset Y$ в $Y$ и в $X$ равносильны.
    \begin{proof} \thmslashn
    
        $Y \implies X$:

        Пусть $G_{\alpha} \subset X$ - открытое покрытие $K$ в $X$.
        
        Тогда $U_{\alpha} = G_{\alpha}\cap Y$ - открытое покрытие $K$ в $Y$. 
        
        Можем выбрать конечное $U_{\alpha_{k}}$.
        
        $U_{\alpha_{k}} \subset G_{\alpha_{k}} \implies G_{\alpha_{k}}$ - конечное открытое покрытие.

        $X \implies Y$:

        Пусть $U_{\alpha} \subset Y$ - открытое покрытие $K$ в $Y$.

        Тогда $\exists{G_{\alpha} \text{ открытое в $X$}}\quad U_{\alpha} = G_{\alpha}\cap Y$.

        $U_{\alpha} \subset G_{\alpha} \implies G_{\alpha}$ - открытое покрытие $K$ в $X$.

        Значит, можем выбрать конечное $G_{\alpha_{k}}$. Тогда 
        \[ \bigcup_{k = 1}^{n} U_{\alpha_{k}} = \bigcup_{k = 1}^{n} (G_{\alpha_{k}} \subset Y) = Y\cap \bigcup_{k=1}^{n}G_{\alpha_{k}} \supset Y\cap K = K .\]

        Значит, $U_{\alpha_{k}}$ - конечное покрытие $K$ в $Y$.
    \end{proof}
\end{theorem}
\begin{theorem} \thmslashn

    Пусть $\left<X, \rho\right>$ - метрическое пространство, $K$ - компакт. Тогда

    \begin{enumerate}
        \item $K$ - замкнуто
            \begin{proof} \thmslashn
            
                Возьмём $a\in X \setminus K$.

                Заметим, что $\forall{x\in K}\quad B_{\frac{\rho(x, a)}{2}}(a) \cap  B_{\frac{\rho(x, a)}{2}}(x) = \emptyset$.

                Возьмём открытое покрытие $K$: $K \subset \bigcup\limits_{x\in K} B_{\frac{\rho(x, a)}{2}}(x)$.

                Выберем конечное: $K \subset \bigcup\limits_{k = 1}^{n} B_{\frac{\rho(a, x_{k})}{2}}(x_{k})$.

                Тогда, при $r := \min\limits_{k} \{\frac{\rho(x_{k}, a)}{2}\}$, $B_{r}(a)\cap K = \emptyset \implies B_{r}(a) \subset X \setminus K \implies a\in \Int (X \setminus K) \implies X \setminus K$ открыто $\implies$ $K$ замкнуто.
            \end{proof}
        \item $K$ - ограничено
            \begin{proof} \thmslashn
            
                Возьмём $a\in K$.

                Тогда $\bigcup_{n = 1}^{\infty} B_{n}(a)$ - открытое покрытие.

                Выберем конечное: $K \subset \bigcup_{k=1}^{m} B_{n_{k}}(a) = B_{r}(a) $, $r := \max\limits_{k} \{n_{k}\} $.
            \end{proof}
    \end{enumerate}
\end{theorem}
\begin{consequence} \thmslashn

    Если $K$ - компакт и $\tilde{K} \subset K$ - замкнуто, то $\tilde{K}$ - компакт.
    \begin{proof} \thmslashn
    
        Пусть $U_{\alpha}$ - открытое покрытие $\tilde{K}$.

        Тогда, если добавить к нему $X \setminus \tilde{K}$ (которое открыто так-как $\tilde{K}$ замкнуто), получится открытое покрытие $K$. Выберем конечное.

        \[ \bigcup_{k=1}^{n} U_{\alpha_{k}} \cup (X \setminus \tilde{K}) \supset K \supset \tilde{K} \implies \bigcup_{k=1}^{n} U_{\alpha_{k}} \supset \tilde{K} \qedhere .\] 
    \end{proof}
\end{consequence}
