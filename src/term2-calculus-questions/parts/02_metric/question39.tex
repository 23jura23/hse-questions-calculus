\Subsection{Билет 39: Связность и линейная связность. Теорема Больцано–Коши. Связность отрезка и линейно связного множества.}

\begin{definition} \thmslashn 

    Пусть $\left<X, \rho\right>$ - метрическое пространство, $A \subset X$.

    $A$ называется связным, если для всех открытых непересекающихся $U, V \subset X$ верно
    \[ A \subset U \cup V \implies \bcases{A \subset U\cr A \subset V} .\]

    (Неформально: $A$ нельзя разбить на два открытых непересекающихся множества)
\end{definition}
\begin{theorem} \thmslashn

    Непрерывный образ связного множества связен
    \begin{proof} \thmslashn
    
        Пусть $f$ - непрерывная функция, $E$ - связное множество.

        Пусть $f(E) \subset U \cup V$, где $U, V$ - открытые непересекающиеся.

        Тогда $f^{-1}(U), f^{-1}(V)$ - \hyperref[cont:preimage_of_open]{открытые} непересекающиеся. И при этом, $E \subset f^{-1}(U) \cup f^{-1}(V)$. 

        Тогда $E$ подмножество одного из них, и $f(E)$ подмножество соответсвующего образа.
    \end{proof}
\end{theorem}
\begin{theorem}[Больцано-Коши] \thmslashn

    Пусть $\left<X, \rho\right>$ - метрическое пространство, $E \subset X$ связно, $f : E \mapsto \R$, $a, b\in E$.

    $A := f(a)$
    $B := f(b)$

    Тогда, $\forall{A < C < B}\quad \exists{c\in E}\quad f(c) = C$.
    \begin{proof} \thmslashn
    
        $U := (-\infty, C)$, $V := (C, \infty)$ - открытые непересекающиеся.
        
        Предположим что такого $c$ не существует. Тогда $C \not\in f(E)$.
        
        Тогда $f(E) \subset U \cup V$. Но при этом $A\in U$, $B\in V$, значит $f(E) \not\subset U$ и $f(E) \not\subset V$, противоречие со связностью $f(E)$ как непрерывного образа связного $E$.
    \end{proof}
\end{theorem}
\begin{theorem} \thmslashn

    Отрезок связен.
    \begin{proof} \thmslashn
    
        Пусть $[a, b] \subset U \cup V$. Без ограничения общности, $b\in V$.

        Предположим что $S := [a, b]\cap U \neq \emptyset$.

        $s := \sup S$.

        Если $s\in V$:
        
        \[ s\in V \overset{\text{открытость}}\implies \exists{\eps > 0}\quad (s-\eps, s+\eps) \subset V \implies (s-\eps, s]\cap U = \emptyset \implies \sup S \le s - \eps < s.\]

        Если $s\in U \implies s\in S$, то $s \neq b$:

        \[ s\in U \implies \exists{\eps}\quad (s-\eps, s+\eps) \subset U\cap [a, b] \implies \sup S \ge s+\eps > s .\]

        ($\exists{\eps}\quad (s - \eps, s+\eps) \subset U$ так-как $U$ открыто, и всегда можем ещё уменьшить чтобы получилось $(s - \eps, s + \eps) \subset [a, b]$).

        В обоих случаях получили противоречие, значит $S = \emptyset \implies [a, b]  \subset V$. Значит, отрезок связен.
    \end{proof}
\end{theorem}
\begin{consequence} \thmslashn

    Носитель пути - связное множество.

    \begin{proof} \thmslashn
    
        Отрезок - связное множество, носитель пути - непрерывный образ отрезка.
    \end{proof}
\end{consequence}
\begin{definition} \thmslashn 

    Пусть $\left<X, \rho\right>$ - метрическое пространство, $A \subset X$.

    $A$ называется линейно связным, если $\forall{x, y\in A}\quad \exists{\gamma : [a, b] \mapsto A}\quad \begin{cases} \gamma(a) = x\\ \gamma(b) = y\\ \gamma \text{ - путь} \end{cases}$,
\end{definition}
\begin{theorem} \thmslashn

    Линейно связное множество связно

    \begin{proof} \thmslashn
    
        Пусть нет, $A \subset U \cup V$ - открытые непересекающиеся, $A\cap U, A\cap V \neq \emptyset $.

        Возьмём $x\in A\cap U$, $y\in A\cap V$.

        Возьмём $\gamma$ - путь от $x$ до $y$.

        $\gamma([a, b]) \subset U \cup V$, $\gamma(a)\in U$, $\gamma(b)\in V$. Противоречие со связностью носителя.
    \end{proof}
\end{theorem}
\begin{definition} \thmslashn 

    Область - открытое линейно связное множество.
\end{definition}
\begin{remark} \thmslashn

    Если $U$ открыто, то $U$ связно $\iff U$ линейно связно. (без доказательства)
\end{remark}
