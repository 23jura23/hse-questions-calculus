\Subsection{Билет 16: ! Свойства замыкания. Предельные точки. Связь с замыканием множества.}

\begin{properties} \thmslashn

    \begin{enumerate}
        \item $A \subset \Cl A$
        \item $\Cl A$ - замкнутое множество
            \begin{proof} \thmslashn
            
                По определению, $\Cl A$ - пересечение замкнутых множетв.
            \end{proof}
        \item $\Cl A = A \iff A $ замкнуто
            \begin{proof} \thmslashn
            
                \begin{equation*}
                    \begin{split}
                        A = \Cl A
                        &\iff X \setminus A = X \setminus \Cl A\\
                        &\iff X \setminus A = \Int (X \setminus A)\\
                        &\iff X \setminus A \text{ открыто}\\
                        &\iff A \text{ замкнуто} \qedhere
                    \end{split}
                \end{equation*}
            \end{proof}
        \item $A \subset B \implies \Cl A \subset \Cl B$
            \begin{proof} \thmslashn

                \begin{equation*}
                    \begin{split}
                        A \subset B 
                        &\implies (X \setminus B) \subset (X \setminus A)\\
                        &\implies \Int (X \setminus B) \subset \Int (X \setminus A)\\
                        &\implies X \setminus \Int (X \setminus A) \subset X \setminus \Int (X \setminus B)\\
                        &\implies \Cl A \subset \Cl B \qedhere
                    \end{split}
                \end{equation*} 
            \end{proof}
        \item $\Cl (A \cup B) = \Cl A \cup \Cl B$
            \begin{proof} \thmslashn
            
                \begin{equation*}
                    \begin{split}
                        \Cl (A \cup B)
                        &= X \setminus \Int (X \setminus (A \cup B)\\
                        &= X \setminus \Int ((X \setminus A)\cap (X \setminus B))\\
                        &= X \setminus (\Int (X \setminus A)\cap \Int (X \setminus B))\\
                        &= (X \setminus \Int (X \setminus A)) \cup (X \setminus \Int (X \setminus B))\\
                        &= \Cl A \cup \Cl B \qedhere
                    \end{split}
                \end{equation*}
            \end{proof}
        \item $\Cl (\Cl A) = \Cl A$
             \begin{proof} \thmslashn
            
                $\Cl A$ замкнуто по свойству $2$, равенство следует из свойства $3$.
            \end{proof}
    \end{enumerate}
\end{properties}
\begin{theorem} \thmslashn

    Пусть $\left<X, \rho\right>$ - метрическое пространство, $A \subset X$.

    \[ a\in \Cl A \iff \forall{r > 0}\quad B_{r}(a)\cap A \neq \emptyset .\]

    \begin{proof} \thmslashn
    
        Необходимость ($\implies$ ):

        Предположим что $\exists{r > 0}\quad B_{r}(a)\cap A = \emptyset$.

        Тогда $a \not\in A$ и $B_{r}(a) \subset X \setminus A$, значит $a\in \Int (X \setminus A) \implies a \not\in X \setminus \Int(X \setminus A) \implies a\not\in \Cl A$.

        Достаточность ($\impliedby$):

        Пусть $a \not\in \Cl A$, тогда $\exists{F}\quad $ - замкнутое надмножество $A$, такое, что $a \not\in F \implies a\in X \setminus F$. При этом, $X \setminus F$ открыто.

        Тогда $\exists{r > 0}\quad B_{r}(a) \subset X \setminus F \subset X \setminus A$.

        Но тогда $B_{r}(a)\cap A = \emptyset$.
    \end{proof}
\end{theorem}
\begin{consequence} \thmslashn

    Пусть $\left<X, \rho\right>$ - метрическое пространство, $A \subset X$, а $U \subset X$ - открытое множетсво. При этом $A\cap U = \emptyset$.

    Тогда $\Cl A\cap U = \emptyset$
    \begin{proof} \thmslashn
    
        \begin{equation*}
            \begin{split}
                x\in \Cl A\cap U 
                &\implies x\in U\\
                &\implies \exists{r > 0}\quad B_{r}(x) \subset U\\
                &\implies B_{r}(x)\cap A \subset U\cap A = \emptyset\\
                &\implies x \not\in \Cl A\\
                &\implies x \not\in \Cl A\cap U
            \end{split}
        \end{equation*}
        
        Получили противоречие, значит таких $x$ не существует.
    \end{proof}
\end{consequence}
\begin{definition} \thmslashn 

    Пусть $\left<X, \rho\right>$ - метрическое пространство.

    Проколотой окрестностью радиуса $r\in \R_{>0}$ с центров в $a\in X$ называется $\punct{B}_{r}(a) := B_{r}(a) \setminus \{a\} = \{x\in X \ssep 0 < \rho(x, a) < r\}$. 
\end{definition}
\begin{definition} \thmslashn 

    Пусть $\left<X, \rho\right>$ - метрическое пространство, $A \subset X$.

    $a\in X$ называется предельной точкой, если $\forall{r > 0}\quad \punct{B}_{r}(a)\cap A \neq \emptyset$.

    Множества предельных точек множества $A$ обозначается $A'$.
\end{definition}
\begin{properties} \thmslashn

    \begin{enumerate}
        \item $\Cl A = A \cup A'$ 
            \begin{proof} \thmslashn
            
                \begin{equation*}
                    \begin{split}
                        a\in \Cl A 
                        &\iff \forall{r > 0}\quad B_{a}(a)\cap A \neq \emptyset\\
                        &\iff \bcases{a\in A \cr \punct{B}_{r}(a)\cap A \neq \emptyset}\\
                        &\iff \bcases{a\in A \cr a\in A'} \qedhere
                    \end{split}
                \end{equation*}
            \end{proof}
        \item $A \subset B \implies A' \subset B'$ 
            \begin{proof} \thmslashn
            
                \begin{equation*}
                    \begin{split}
                        a\in A'
                        &\implies \forall{r}\quad \punct{B}_{r}(a)\cap A \neq \emptyset\\
                        &\implies \punct{B}_{r}(a)\cap B \neq \emptyset\\
                        &\implies a\in B'
                    \end{split}
                \end{equation*}
            \end{proof}
        \item $(A \cup B)' = A' \cup B'$
            \begin{proof} \thmslashn
            
                \begin{equation*}
                    \begin{split}
                        A \subset A \cup B &\implies A' \subset (A \cup B)'\\
                        B \subset A \cup B &\implies B' \subset (A \cup B)'\\
                                           &\implies A' \cup B' \subset (A \cup B)'
                    \end{split}
                \end{equation*}

                Покажем другое включение: возьмём $x\in (A \cup B)'$.

                Пусть $x \not\in A'$: Тогда $\exists{R > 0}\quad \punct{B}_{R}(x)\cap A = \emptyset$. 

                Заметим, что $\forall{0 < r \le  R}\quad \punct{B}_{r}(x)\cap A \subset \punct{B}_{R}(x)\cap A  = \emptyset$, значит $\forall{r > 0}\quad \exists{0 < R_{r} < r}\quad \punct{B}_{R_{r}}(x)\cap A = \emptyset$ (если $r < R$, то подойдёт он сам, иначе подойдёт  $R - \eps$).

                Так-как $\punct{B}_{R_{r}}(x)\cap (A \cup B) \neq \emptyset$, значит $\punct{B}_{R_{r}}(x)\cap B \neq \emptyset$ (если пересекаемся с объединением, но не с $A$, значит с  $B$).
                Тогда
                \[ \forall{r > 0}\quad \punct{B}_{r}(x)\cap B \supset \punct{B}_{R_{r}}(x)\cap B \neq \emptyset \text{ вложение так-как } r > R_{r} .\]

                Значит, $x\in B'$
            \end{proof}
        \item $A' \subset A \iff A$ - замкнутое
            \begin{proof} \thmslashn
            
                \begin{equation*}
                    \begin{split}
                        A \text{ - замкнутое}
                        &\iff A = \Cl A\\
                        &\iff A = A \cup A'\\
                        &\iff A' \subset A \qedhere
                    \end{split}
                \end{equation*}
            \end{proof}
    
    \end{enumerate}
\end{properties}
\begin{theorem} \thmslashn

    Пусть $\left<X, \rho\right>$ - метрическое пространство, $A \subset X$.

    \[ a\in A' \iff \forall{r > 0}\quad B_{r}(a)\cap A \text{ содержит бесконечно много точек} .\]
    \begin{proof} \thmslashn
    
        Необходимость ($\implies$):
        
        Знаем, что $\punct{B}_{r}(a)\cap A \neq \emptyset$, возьмём точку $x_1\in \punct{B}_{r}(a)\cap A$, возьмём $r_2 = \rho(x_1, a)$, знаем, что $\punct{B}_{r}(a)\cap A \neq \emptyset$, можем взять точку оттуда, и вообще повторять бесконечное число раз.

        Достаточность ($\le$):
        $B_{r}(a)\cap A$ содержит бесконечно много точек $\implies \punct{B}_{r}(a)\cap A$ содержит бесконечно много точек $\implies \punct{B}_{r}(a)\cap A \neq \emptyset \implies a\in A'$.
    \end{proof}
\end{theorem}
