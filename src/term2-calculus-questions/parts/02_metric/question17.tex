\Subsection{Билет 17: Индуцированная метрика. Открытые и замкнутые множества в пространстве и в подпространстве.}

\begin{definition} \thmslashn 

    Пусть $\left<X, \rho\right>$ - метрическое пространство, $Y \subset X$.

    Тогда пара $\left<Y, \left. \rho\right|_{Y \times Y}\right>$ называется метрическим подпростраством $X$.

    Далее, при разговое о подпростравах обычно будет указываться только множество, а метрика использоваться та-же что и для основного пространства.
\end{definition}

\begin{theorem} \thmslashn

    Пусть $\left<X, \rho\right>$ - метрическое пространство, $Y$ - его подпространство.

    $A \subset Y$ открыто в $Y$ тогда и только тогда, когда $\exists{G}$ открытое в $X$, такое, что $A = G\cap Y$
    \begin{proof} \thmslashn
    
        Необходимость ($\implies$):
        
        \begin{equation*}
            \begin{split}
                A \text{ - открыто в $Y$} 
                &\implies \forall{a\in A}\quad \exists{r_{a} > 0}\quad B^{Y}_{r_{a}}(a) \subset A\\
                &\implies A = \bigcup_{a\in A} B^{Y}_{r(a)}(A) \subset \bigcup_{a\in A} B^{X}_{r(a)}(a) =: G
            \end{split}
        \end{equation*}
        $G$ - подходящее множество - оно открыто как объединение открытых, покажем что $A = G\cap Y$:
        \[ B^{Y}_{r}(x) = B^{X}_{r}(x)\cap Y .\]
        \[ G\cap Y = Y\cap \bigcup_{a\in A} B^{X}_{r(a)}(a) = \bigcup_{a\in A} B^{Y}_{r(a)}(a) = A .\]

        Достаточность ($\impliedby$):

        Пусть $A = G\cap Y$. Возьмём $a\in A$.

        \begin{equation*}
            \begin{split}
                G \text{ открыто в $X$} 
                &\implies \exists{r > 0}\quad B^{X}_{r}(a) \subset G\\
                &\implies B^{X}_{r}(a)\cap Y \subset G\cap Y\\
                &\implies B^{Y}_{r}(a) \subset A\\
                &\implies A \text{ открыто в $Y$} \qedhere
            \end{split}
        \end{equation*}
    \end{proof}
\end{theorem}
\begin{theorem} \thmslashn

    Пусть $\left<X, \rho\right>$ - метрическое пространство, $Y$ - его подпространство.

    $A \subset Y$ замкнуто тогда и только тогда, когда $\exists{F}$ замкнутое в $X$, такое, что $A = F\cap Y$.
    \begin{proof} \thmslashn
    
        $F := X \setminus G$, где $G$ - открытое в  $X$ такое, что  $G\cap Y = Y \setminus A$ существование которого экивалентно открытости $Y \setminus A \iff $ замкнутости $A$.
        \begin{equation*}
            \begin{split}
                F\cap Y 
                &= (X \setminus G)\cap Y\\
                &= (X\cap Y) \setminus G\\
                &= Y \setminus G\\
                &= Y \setminus (G\cap Y)\\
                &= Y \setminus (Y \setminus A)\\
                &= A
            \end{split}
        \end{equation*}
    \end{proof}
\end{theorem}
\begin{example} \thmslashn

    Пусть $X = \R$, $Y = [0, 2)$.

    $[0, 1)$ открыто в $Y$, например $[0, 1) = (-1, 1)\cap Y$.

    При $r \le 2$, $B_{r}^{Y}(0) = [0, r)$.

    $[1, 2)$ замкнуто в $Y$, например $[1, 2) = [1, 2]\cap Y$.
\end{example}
