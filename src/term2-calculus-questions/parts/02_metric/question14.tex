\Subsection{Билет 14: Внутренние точки и внутренность множества. Свойства.}

\begin{definition}[повтор] \thmslashn 

    Пусть $\left<X, \rho\right>$ - метрическое пространство, $A \subset X$.

    Точка $a\in A$ называется внутренней если $\exists{r > 0}\quad B_{r}(a) \subset A$.

    Множество внутренних точек называется внутренностью множества, и обозначается $\Int A$.
\end{definition}

\begin{properties} \thmslashn

    Пусть $\left<X, \rho\right>$ - метрическое пространство, $A \subset X$.

    \begin{enumerate}
        \item $\Int A \subset A$
        \item $\Int A$ - объеденение всех открытых множеств содержащихся в $A$.
            \begin{proof} \thmslashn
            
                Пусть $G = \bigcup\limits_{\alpha\in I} U_{\alpha}$, где $U_{\alpha} \subset A$ - открытое.

                $G \subset \Int A$:
                \begin{equation*}
                    \begin{split}
                        x\in G 
                        &\implies \exists{\alpha\in I}\quad x\in U_{\alpha}\\
                        &\implies \exists{r > 0}\quad B_{r}(x) \subset U_{\alpha} \subset A\\
                        &\implies x\in \Int A
                    \end{split}
                \end{equation*}
                $\Int A \subset G$:  $x\in \Int A \implies \exists{r > 0}\quad B_{r}(x) \subset A$. $B_{r}(x)$ - открытое множество, значит $\exists{\alpha\in I}\quad U_{\alpha} = B_{r}(x) \implies x\in G$.
            \end{proof}
        \item $\Int A$ - откртое множество
            \begin{proof} \thmslashn
            
                $A$ - объединение открытых множеств, значит открыто.
            \end{proof}
        \item $\Int A = A \iff A$ - открыто
            \begin{proof} \thmslashn
            
                Необходимость ($\implies$): $\Int A$ открыто.

                Достаточность ($\impliedby$): $A$ открыто $\implies$ все точки внутренние $\implies$ $A = \Int A$.
            \end{proof}
        \item $A \subset B \implies \Int A \subset \Int B$
        \item $\Int (A\cap B) = \Int A\cap \Int B$
            \begin{proof} \thmslashn

                В сторону $ \subset $:
                \begin{equation*}
                    \begin{split}
                        A\cap B \subset A 
                        &\implies \Int (A\cap B) \subset \Int A\\
                        A\cap B \subset B 
                        &\implies \Int (A\cap B) \subset \Int B\\
                        &\implies \Int (A\cap B) \subset \Int A\cap \Int B
                    \end{split}
                \end{equation*}
                В сторону $\supset$:
                \begin{equation*}
                    x\in \Int A\cap \Int B \Rightarrow
                    \left\{
                        \begin{split}
                            x&\in \Int A \Rightarrow \exists{r_1}\quad B_{r_1}(x) \subset A\\
                            x&\in \Int B \Rightarrow \exists{r_2}\quad B_{r_2}(x) \subset B\\
                        \end{split}
                    \right\} \Rightarrow
                    B_{\min \{r_1, r_2\} }(x) \subset A\cap B  \Rightarrow
                    x\in \Int (A\cap B)
                \end{equation*}
            \end{proof}
        \item $\Int \Int A = \Int A$
             \begin{proof} \thmslashn
            
                Заметим, что $\Int A$ - открытое по $3$, дальше по $4$ видно равенство.
            \end{proof}
    \end{enumerate}
\end{properties}
