\Subsection{Билет 29: ! Непрерывный образ компакта. Теорема Вейерштрасса. Непрерывность обратного отображения.}

\begin{theorem} \thmslashn

    Пусть $\left<X, \rho_{X}\right>$, $\left<Y, \rho_{Y}\right>$ - метрические пространства, $f : X \mapsto Y$, $f$ непрерывна, $K \subset X$ - компакт.

    Тогда $f(K)$ компакт.

    \begin{proof} \thmslashn
    
        Возьмём открытое покрытие $f(K)$, назовём его $U_{\alpha}$.

        Тогда $V_{\alpha} = f^{-1}(U_{\alpha})$ - открытое покрытие $K$.

        Выберем конечное $V_{\alpha_{k}}$.

        Тогда $K \subset \bigcup\limits_{k=1}^{n} V_{\alpha_{k}} \implies f(K) \subset \bigcup\limits_{k=1}^{n} f(V_{\alpha_{k}}) = \bigcup\limits_{k=1}^{n} U_{\alpha_{k}}   $.
    \end{proof}
\end{theorem}
\begin{theorem}[Вейерштрасса] \thmslashn

    Пусть $\left<X, \rho_{X}\right>$ - метрическое пространство, $f : X \mapsto \R$, $f$ непрерывна, $K \subset X$ - компакт.

    Тогда $\exists{u, v\in K}\quad \forall{x\in K}\quad f(u) \le f(x) \le f(v)$.
    \begin{proof} \thmslashn
    
        $f(K)$ - компакт $\implies$ замкнут и ограничен.

        Ограничен $\implies$ $\inf f$ и $\sup f$ - конечные.

        Предположим что $b := \sup f \not\in f(K)$.

        Тогда можем взять последовательность $x_{n}\in f(K)$, $x_{n} \to b$. Тогда $b$ - предельная точка $f(K)$. $b\in f(K)' \subset \Cl f(K) = f(K)$. Противоречие. Значит $b\in f(K) \implies \exists{v\in K}\quad f(v) = b$. Аналогично для $\inf f$.
    \end{proof}
\end{theorem}

\begin{theorem} \thmslashn

    Пусть $\left<X, \rho_{X}\right>$, $\left<Y, \rho_{Y}\right>$ - метрические пространства, $f : X \mapsto Y$, $f$ непрерывная биекция, $X$ - компакт.

    Тогда $f^{-1}$ непрерывна.
    \begin{proof} \thmslashn
    
        Пусть $g := f^{-1}$.

        Пусть $U \subset X$ - открытое множество.

        Заметим, что $f(U) = Y \setminus f(X \setminus U)$ (так-как биекция).

        $X \setminus U$ - замкнутое подмножество компакт $\implies$ компакт $\implies$ $f(X \setminus U)$ замкнуто $\implies$ $Y \setminus f(X \setminus U)$ - открыто.

        $f(U) = g^{-1}(U)$, значит для $g$ прообраз открытого открыт $\implies$ $g$ непрерывно.
    \end{proof}
\end{theorem}
