\Subsection{Билет 24: Секвенциальная компактность. Компактность и предельные точки. Секвенциальная компактность компакта.}

\begin{definition} \thmslashn 

    Пусть $\left<X, \rho\right>$ - метрическое пространство.

    $K \subset X$ называется секвенциально компактным, если из любой последовательности точек из $K$ можно выбрать подпоследовательность сходящуюся к точке из $K$.
\end{definition}
\begin{theorem} \thmslashn

    Пусть $\left<X, \rho\right>$ - метрическое пространство, $K \subset X$ секвенциально компактно.

    Тогда всякое бесконечное множество точек из $K$ имеет хотя-бы одну предельную точку в $K$.
    \begin{proof} \thmslashn
    
        Выберем последовательность $x_{n}$ из этого подмножества, $x_{n}\in K$, значит можем выбрать сходящуюся подпоследовательность, а сходится она может только к предельной точке.
    \end{proof}
\end{theorem}
\begin{theorem} \thmslashn

    Пусть $\left<X, \rho\right>$ - метрическое пространство, $K \subset X$ - компакт.

    Тогда $K$ секвенциально компактно.
    \begin{proof} \thmslashn
    
        Пусть $x_{n}\in K$ - последовательность. $D = \{x_{n}\}$ (множество элементов).

        Если $D$ конечно, то какая-то точка встречается в последовательности бесконечное количество раз, выберем подпоследовательность состояющую только из этой точки, она сходится.

        Заметим, что в $D$ обязательно есть предельная точка:

        Пусть нету. Тогда $D = D \cup \emptyset = D \cup D' = \Cl D \implies D$ замкнуто. Замкнутое подмножество компакт - компакт.

        Так-как $\forall{n}\quad x_{n}$ не предельная в $D$, можем выбрать $r_{n}$, такие, что $\punct{B_{r_{n}}}(x_{n})\cap D = \emptyset \implies B_{r_{n}}(x_{n})\cap D = \{x_{n}\} $.

        Покроем $D$ такими шарами. Каждый шар покрывает ровно одну точку и точек бесконечно $\implies$ нельзя выбрать конечное покрытие. Противоречие.

        Значит, $\exists{a\in D'}$.

        Возьмём произвольную точку из последовательности $x_{n_1}$. Пусть $r_{k} := \min \{\frac{1}{k}, \min\limits_{n < k} \{x_{n}\} \} $.

        Будем брать $x_{n_{k}}$ как произвольную точку из $\punct{B}_{r_{k-1}}(a)$. Так-как он ближе к $a$ чем все предыдущие, $n_{k} > n_{k-1}$, значит получится подпоследовательность.

        При этом, $\rho(x_{n_{k}}, a) < \frac{1}{k - 1} \implies \lim\limits_{k \to \infty} x_{n_{k}} = a$. При этом, $D \subset K \implies \Cl D \subset \Cl K = K$. А $a\in D' \subset \Cl D \subset K \implies a\in K$.
    \end{proof}
\end{theorem}
