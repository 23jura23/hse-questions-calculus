\Subsection{Билет 13: Открытые множества: определение и свойства.}

\begin{definition} \thmslashn 

    Пусть $\left<X, \rho\right>$ - метрическое пространство, $A \subset X$.

    Точка $a\in A$ называется внутренней если $\exists{r > 0}\quad B_{r}(a) \subset A$.

    Множество внутренних точек называется внутренностью множества, и обозначается $\Int A$.
\end{definition}
\begin{definition} \thmslashn 

    Пусть $\left<X, \rho\right>$ - метрическое пространство, $A \subset X$.

    $A$ называется открытым, если все его точки внутренние.
\end{definition}
\begin{properties} \thmslashn

    Пусть $\left<X, \rho\right>$ - метрическое пространство.
    \begin{enumerate}
        \item $ \emptyset$, $X$ - открытые множества.
        \item Объединение любого количества открытых множеств открыто
            \begin{proof} \thmslashn
            
                Пусть $\forall{\alpha\in I}\quad A_{\alpha}$ - открытое множество. $A := \bigcup\limits_{\alpha\in I} A_{\alpha}$.
               Возьмём точку $a$,  $\exists{\beta\in I}\quad a\in A_{\beta}$.

               Так-как $A_{\beta}$ открытое, $\exists{r > 0}\quad B_{r}(a) \subset A_{\beta} \subset A$. 
            \end{proof}
        \item Пересечение конечного количества открытых множеств открыто
            \begin{proof} \thmslashn
            
                Пусть $I = [1; n]$, $\forall{k\in I}\quad a\in A_{k}$, $A_{k}$ - открытое.

                Тогда $\forall{k\in I}\quad \exists{r_{k} > 0}\quad B_{r_{k}}(a) \subset A_{k}$.

                Пусть $r = \min\limits_{k} r_{k} > 0$.

                Тогда $\forall{k\in I}\quad B_{r}(a) \subset B_{r_{k}}(a) \subset A_{k} \implies B_{r}(a) \subset \bigcap\limits_{k=1}^{n} A_{k}$.
            \end{proof}
            \begin{remark} \thmslashn
            
                Конечность важна.

                Возьмём последовательность $U_{n} = B_{\frac{1}{n}}(a)$. Все эти множества открытые, но их пересечение- $\{a\}$ - не обязательно открыто. (Например, $X=\R$, $a=0$. $\{0\}$ не открыто).
            
            \end{remark}
        \item $\forall{a\in X}\quad \forall{r\in \mathbb{R}}\quad B_{r}(a)$ - открытое множество.
            \begin{proof} \thmslashn
                

                Пусть $x\in B_{r}(a)$, $\tilde{r} = r - \rho(x, a)$.

                Покажем что $B_{\tilde{r}}(x) \subset B_{r}(a)$:
                \begin{equation*}
                    \begin{split}
                        y\in B_{\tilde{r}}(x) 
                        &\implies \rho(y, x) < \tilde{r}\\
                        &\implies \rho(y, x) < r - \rho(x, a)\\
                        &\implies \rho(y, x) + \rho(x, a) < r\\
                        & \overset{\triangle}{\implies} \rho(y, a) < r\\
                        &\implies y\in B_{r}(a) \qedhere
                    \end{split}
                \end{equation*}
            \end{proof}
    \end{enumerate}
\end{properties}
