\Subsection{Билет 26: $\eps$-сети и вполне ограниченность. Свойства. Связь с компактностью (теорема Хаусдорфа). Теорема о характеристике компактов в $\R^{d}$. Теорема Больцано-Вейерштрасса.}

\TODO{Сети и Хаусдорфа опять не видно не у меня не у Ани.}

\begin{definition} \thmslashn 

    Пусть $a, b\in \R^{d}$.

    Замкнутый параллелепипед: $[a, b] = [a_1, b_1] \times [a_2, b_2] \times\ldots \times [a_{d}, b_{d}]$.

    Открытый параллелепипед: $(a, b) = (a_1, b_1) \times (a_2, b_2) \times \ldots \times (a_{d}, b_{d})$.
\end{definition}

\begin{theorem}[О вложенных параллелепипедах] \thmslashn

    Пусть $P_1 \supset P_2 \supset P_3 \supset \ldots$ - замкнутые параллелепипеды.

    Тогда $\bigcap\limits_{n=1}^{\infty} P_{n} \neq \emptyset $.

    \begin{proof} \thmslashn
    
        Обозначим $P_{n} =: [a^{(n)}, b^{(n)}]$.

        По теореме о вложенных отрезках:
        \[ \forall{k\in [1, n]}\quad \exists{c_{k}\in \bigcap_{n=1}^{\infty} [a_{k}^{(n)}, b_{k}^{(n)}] }\quad  .\] 

        Тогда, $c = (\forall{n}\quad c_1, \ldots, c_{d})\in P_{n}$
    \end{proof}
\end{theorem}

\begin{theorem} \thmslashn

    Замкнутый куб (замкнутый параллелепипед, все координаты углов которого равны для данного угла) в $\R^{d}$ - компакт.

    \begin{proof} \thmslashn
    
        Пусть $K$ - замкнутый куб и $U_{\alpha}$ - его открытое покрытие. Предположим что выбрать конечное нельзя.

        Разобьём $K$ на $2^{d}$ кубов, со стороной равной половине стороны $K$. $U_{\alpha}$ - открытое покрытие каждого такого куба.

        Хотя-бы один маленький куб нельзя будет покрыть конечным покрытием, назовём его $K_1$, повторим для него, получим последовательность $K_1 \supset K_2 \supset \ldots$.

        По теореме о вложенных параллелепипедах, $\exists{c\in \bigcap\limits_{n=1}^{\infty} K_{n} }\quad $. 

        $\exists{\alpha_0}\quad c\in U_{\alpha_0}$, $U_{\alpha_0}$ открытое $\implies \exists{r > 0}\quad B_{r}(c) \subset U_{\alpha_0}$.

        Заметим, что длина ребра $K_{n} = \frac{l}{2^{n}} \to 0$ ($l$ - длина ребра $K$) $\implies$ максимальное расстояние между точками - $\sqrt{d} \frac{l}{2^{n}} \to 0$ (какой-то факт о евклидовой метрике).

        Тогда, $\exists{n}\quad \sqrt{d} \frac{l}{2^{n}} < r $. Значит, $\exists{n}\quad K_{n} \subset B_{r}(c) \subset U_{\alpha_0}$. Но это противоречит тому, что для $K_{n}$ нельзя выбрать конечное покрытие. Значит $K$ - компакт.


    \end{proof}
\end{theorem}

\begin{theorem} \thmslashn

    Пусть $K \subset \R^{d}$ с евклидовой метрикой. Тогда следующие услвия эквивалентны:
    \begin{enumerate}
        \item $K$ - компакт
        \item $K$ - замкнуто и ограничено
        \item $K$ секвенциально компактно.
    \end{enumerate}
    \begin{proof} \thmslashn
    
        $1 \implies 2$ и $1 \implies 3$ уже были.

        $2 \implies 1$: $K$ ограничено $\implies K \subset B_{r}(a) \subset $ куб. $K$ - замкнутое подмножество компакта $\implies$ $K$ - компакт.

        $3 \implies 2$:

        Пусть $K$ не замкнуто. Тогда есть предельная точка не в $K$. Можем выбрать сходящуюся к ней последовательность, но тогда любая подполседовательность сходится к ней $\implies$ не можем выбрать сходящуюся к точке из $K$. Противоречие $\implies K$ замкнуто.

        Пусть $K$ не ограничено $\implies \forall{n > 0}\quad K \not \subset B_{n}(0)$.

        Тогда, можем выбрать последовательность вида $x_{n}\in K \setminus B_{n}(0)$. Тогда $\rho(0, x_{n}) \ge n$.

        Выберем сходящуюся к $a\in K$ подпоследовательность $x_{n_{k}}$. Тогда $x_{n_{k}}$ ограничена, причём ограничивающий шар с центром в $a$ точно существует: $x_{n_{k}}\in B_{r}(a) \implies \rho(x_{n_{k}}, a) < r \overset{\triangle}{\implies} \rho(x_{n_{k}}, 0) < r + \rho(0, a)$. Противоречине, значит $K$ ограничено.
    \end{proof}
\end{theorem}
\begin{remark} \thmslashn

    $3 \implies 1$ верно для произвольного пространства, но доказательство сложное.

    $2 \implies 1$ в общем случае неверно:

    Рассмотрим $\R$ с метрикой лентяя. $[0, 1] \subset B_{2}(0)$, и есть замкнутость.

    Но из $\bigcup_{x\in [0, 1]} B_{\frac{1}{2}}(x)$ нельзя выбрать конечное покрытие, так-как каждый шар содержит лишь одну точку.
\end{remark}
\begin{theorem}[Больцано-Вейерштрасса] \thmslashn

    Из всякой ограниченной последовательности в $\R^{d}$ можно выбрать сходящуюся подпоследовательность.
    \begin{proof} \thmslashn
    
        $\{x_{n}\}$ ограничено $\implies \{x_{n}\} \subset B_{R}(a) \subset \overline{B}_{R}(a) $ - замкнуто и ограничено $\implies$ компакт $\implies$ секвенциально компактно $\implies$ можно выбрать сходящуюся подпоследовательность.
    \end{proof}
\end{theorem}
