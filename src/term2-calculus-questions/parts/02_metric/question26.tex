\Subsection{Билет 26: $\eps$-сети и вполне ограниченность. Свойства. Связь с компактностью (теорема Хаусдорфа). Теорема о характеристике компактов в $\R^{d}$. Теорема Больцано-Вейерштрасса.}

\begin{definition} \thmslashn

  $X$ - метрическое пространство, $A \subset X$. Тогда $E$ - \textit{$\eps$-сеть множества} $A$ если $\forall a \in A \quad \exists e \in E : \rho(a, e) < \eps$.
  
  Также можно переписать в виде: $E$ - $\eps$-сеть множества $A$ если $A \subset \bigcup\limits_{e \in E}{B_{\eps}(e)}$.

  Менее формально $E$ - $\eps$-приближение множества $A$.
\end{definition}

\begin{definition} \thmslashn

  Множество $A$ называется вполне ограниченным, если $\forall \eps > 0 \quad \exists$ \textbf{конечная}  $\eps$-сеть множества $A$.
  
\end{definition}

\begin{definition} \thmslashn 

    Пусть $a, b\in \R^{d}$.

    Замкнутый параллелепипед: $[a, b] = [a_1, b_1] \times [a_2, b_2] \times\ldots \times [a_{d}, b_{d}]$.

    Открытый параллелепипед: $(a, b) = (a_1, b_1) \times (a_2, b_2) \times \ldots \times (a_{d}, b_{d})$.
\end{definition}

\begin{properties} \thmslashn

  \begin{enumerate}
    \item Из вполне ограниченности следует ограниченность. 
      \begin{proof} \thmslashn

        Возьмем $\eps = 1$. Тогда $A \subset \bigcup\limits_{i = 1}^{n}{B_{1}(e_i)} \subset B_{R}(e_1)$, где $R = 1 + \max\{\rho(e_1, e_2), \rho(e_1, e_3), \dots, \rho(e_1, e_n)\}$.
      \end{proof}
    \item В $\R^d$ из ограниченоости следует вполне ограниченность.
      \begin{proof} \thmslashn

        Любое ограниченное множество в $\R^d$ можно впихнуть ограниченный параллелепипед. Возьемем $\delta < \frac{\eps}{\sqrt{d}}$. Разделим параллелелипипед на кубики со стороной $\delta$. А каждый такой кубик покрывается шариком радиуса $\eps$, так как наибольшее расстояние в таком кубике равно $\sqrt{d}\delta$. Получили конечную $\eps$-сеть.
      \end{proof}
  \end{enumerate}
\end{properties}

\begin{theorem}[Хаусдорфа] \thmslashn

  \begin{enumerate}
    \item Компакт вполне ограничен 
    \item Если $X$ - полное метрическое пространство и $K \subset X$, из замкнутости и вполне ограниченности следует компактность.
  \end{enumerate}

  \begin{proof} \thmslashn

    \begin{enumerate}
      \item $K \subset \bigcup\limits_{x \in K}{B_{\eps}(x)}$ - открытое покрытие. Можем выбрать конечное покрытие, в таком случае центры шариков из конечного поркытия будут образовывать конечную $\eps$-сеть.
      \item Проверим секвенциальную компактность. Пусть $\{x_n\} \in K$. Надо доказать, что из нее можно выбрать сходящуюся подпоследовательность, предел которой лежит в $K$. \par
        Рассмотрим $\eps_n = \frac{1}{n}$. 
        Так как $K$ вполне ограниченно возьмем $K \subset \bigcup\limits_{i = 1}^{n}{B_{\eps_1}(e_i)}$.
        Так как шариков коненчное число, то в каком-то из них содержится бесконечное кол-во $x_i$ из $\{x_n\}$. Пусть это шарик $B_{\eps_1}(e_l) = V_1$. \par
      $V_1 \cap K$ - вполне ограничено, значит можно опять покрыть конечным числом шариков : $V_1\cap K \subset \bigcup\limits_{i = 1}^{m}{B_{\eps_2}(e_i)}$. Также существует $B_{\eps_2}(e_k) = V_2$, который содержит бесконенчное число элементов. \par
      Так сделаем для каждого $\eps_i$. \par
      Выпишем таблицу, где в $i$- ой строке стоят элементы из $V_i$. Пусть $a_{ij}$ - элемент таблицы. Тогда рассмотрим элементы $a_{11}, a_{22}, \dots, a_{nn}, \dots$. Эти точки образуют подпоследовательность $\{x_n\}$. Покажем, что это фундаментальная последовательность. \par
      Для этого рассмотрим $\rho(a_{kk}, a_{nn}), k < n$. По построению таблицы элемент $a_{nn}$ также лежит в шарике, который был взят на $k$-ом шагу. Значит $\rho(a_{kk}, a_{nn}) < 2\eps_{k} = \frac{2}{k}$. Значит данная последовательность фундаментальна $\implies$ имеет предел и так как $K$ - замкнуто, данный предел $\in K$. Значит $K$ - секвенциально компактно, а значит и просто компактно. 
    \end{enumerate}
  \end{proof}
  \begin{remark} \thmslashn

    Если бы $K$ было не замкнуто, то предел мог бы не лежать в $K$, поэтому не было бы секвенциальной компактности.
  \end{remark}
\end{theorem}

\begin{theorem}[о характеристике компактов в $\R^d$] \thmslashn

  В $\R^d$ компактность тоже самое, что и замкнутость и ограниченность.
  \begin{proof} \thmslashn

    \begin{itemize}
      \item Из компактности замкнутость и ограниченность. Компактное множество - замкнуто и по теореме Хаусдорфа из компактности следует вполне ограниченность, а из вполне ограниченности следует ограниченность.
      \item Из ограниченности в $\R^d$ следует вполне ограниченность, а так как $\R^d$ - полно, то по теореме Хаусдофра следует, что замкнутое, вполне ограниченное множество -  компактно. 
    \end{itemize}
  \end{proof}
\end{theorem}

\begin{theorem}[Больцано-Вейерштрасса] \thmslashn

    Из всякой ограниченной последовательности в $\R^{d}$ можно выбрать сходящуюся подпоследовательность.
    \begin{proof} \thmslashn
    
        $\{x_{n}\}$ ограничено $\implies \{x_{n}\} \subset B_{R}(a) \subset \overline{B}_{R}(a) $ - замкнуто и ограничено $\implies$ компакт $\implies$ секвенциально компактно $\implies$ можно выбрать сходящуюся подпоследовательность.
    \end{proof}
\end{theorem}


