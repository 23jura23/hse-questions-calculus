\Subsection{Билет 21: ! Фундаментальные последовательности. Свойства. Полнота. Полнота $\R^{d}$}

\textbf{Тут что-то странное с порядком билетов, рекомендуется сначала прочитать билет 22}

\begin{definition} \thmslashn 

    Пусть $\left<X, \rho\right>$ - метрическое пространоство.

    Последовательность $x_{n}$ называется фундаментальной
    \[ \forall{\eps > 0}\quad \exists{N\in \N}\quad \forall{n, m \ge  N}\quad \rho(x_{n}, x_{m}) < \eps  .\] 
\end{definition}

\begin{lemma} \thmslashn

    Фундаментальная последовательность ограничена

    \begin{proof} \thmslashn
    
        Подставим $\eps = 1$, получим $\forall{n \ge N}\quad \rho(x_{N}, x_{n}) < 1 \implies x_{n}\in B_{1}(x_{N})$, пусть 
        \[ r = \max \{1, \max\limits_{k < N} \{\rho(x_{N}, x_{k})\} \} .\]

        Тогда $\forall{n\in \N}\quad x_{n}\in B_{r}(x_{N})$.
    \end{proof}
\end{lemma}

\begin{lemma} \thmslashn

    Пусть $\left<X, \rho\right>$ - метрическое пространство.

    $x_{n}\in X$, $\exists{a\in X}\quad \lim\limits_{n \to \infty} x_{n} = a$.

    Тогда $x_{n}$ фундаментальна.

    \begin{proof} \thmslashn
    
        Из предела получаем следующие:
        \[ \forall{\eps > 0}\quad \exists{N\in \N}\quad \forall{n \ge N}\quad \rho(x_{n}, a) < \frac{\eps}{2} .\]

        Тогда
        \[ \forall{n, m \ge N}\quad \rho(x_{n}, x_{m}) \overset{\triangle}{\le} \rho(x_{n}, a) + \rho(a, x_{m}) < \eps .\] 
    \end{proof}
\end{lemma}

\begin{definition} \thmslashn 

    Метрическое пространство называется полным, если любая фундаментальная последовательность имеет предел.
\end{definition}
\begin{lemma} \thmslashn

    Пусть $\left<X, \rho\right>$ - метрическое пространство.

    Пусть $x_{n}\in X$ - фундаментальна, а $\lim\limits_{k \to \infty} x_{n_{k}} = a$. Тогда $\lim\limits_{n \to \infty} x_{n} = a$.
    \begin{proof} \thmslashn
    
        \[ \lim\limits_{n \to \infty} x_{n_{k}} = a \implies \forall{\eps > 0}\quad \exists{M\in \N}\quad \forall{k \ge M}\quad \rho(x_{n_{k}}, a) < \eps  .\]
        \[ x_{n} \text{ - фундаментальна } \implies \forall{\eps > 0}\quad \exists{N \in \N}\quad \forall{n, m \ge N}\quad \rho(x_{n}, x_{m}) < \eps .\] 

        Пусть $L = \max \{N, M\}$.

        Тогда $\forall{n > L}\quad \exists{k}\quad \rho(x_{n}, a) < \rho(x_{n}, x_{n_{k}}) + \rho(x_{n_{k}}, a) < 2\eps$.

        Значит, $\rho(x_{n}, a) \to 0 \implies x_{n} \to a$.
    \end{proof}
\end{lemma}
\begin{consequence} \thmslashn

    \begin{enumerate}
        \item $\R^{d}$ - полное
            \begin{proof} \thmslashn
            
                Пусть $x_{n}\in \R^{d}$ - фундаментальная последовательность.

                Тогда $x_{n}$ ограничена $\implies$ $\exists{x_{n_{k}}}\quad $ - сходящаяся к точке из $\R^{d}$ подпоследовательность (Больцано-Вейерштрасс из следующего билета), пусть $\lim\limits_{k \to \infty} x_{n_{k}} = a$.


                Тогда $\lim\limits_{n \to \infty} x_{n} = a\in \R^{d}$.
            \end{proof}
        \item $K$ - компакт в $\left<X, \rho\right>$ $\implies \left<K, \rho\right>$ - полное.
            \begin{proof} \thmslashn
            
               $K$ - компакт,  $x_{n}\in K$ - фундаментальна.

               $\exists{x_{n_{k}}\in K}\quad \lim\limits_{k \to \infty} x_{n_{k}} = a\in K \implies \lim\limits_{n \to \infty} x_{n} = a\in K$.
            \end{proof}
    \end{enumerate}
\end{consequence}
